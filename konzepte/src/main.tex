%! Author = Adrian Helberg
%! Date = 09.05.2020

% Preamble
\documentclass{beamer}
\usepackage[ngerman]{babel}
\usepackage[T1]{fontenc}

% Packages
\usepackage{amsmath}

% Document
\begin{document}

    \title{Bachelor Thesis}
    \subtitle{Konzepte und Ideen}
    \author{Adrian Helberg}
    \date{\today}
    \institute{HAW Hamburg}

    \frame{
        \maketitle
    }

    \frame{
        \frametitle{Vorangehendes Projekt}
        \textit{"`Erstellung einer Softwarelösung zur Digitalisierung im Barwesen''}
        \\~\\
        \begin{block}{Ziele und Vision}
            \begin{itemize}
                \item Erstellung einer mobilen Anwendung zum Finden, Verwalten und Bewerten von Bars
                \item Diverse Teilsysteme einzelner Bar-Prozesse (Reservierung, Events, Rezepte, Inventar, \ldots)
                \item Datenanalyse zur Marktanalyse, Kostenanalyse, Trinkverhalten, Expertise (Knowhow), \ldots $\Rightarrow$ als Monetarisierungskonzept
                \item Kommunikation zwischen Bars und Endkunden (\textit{Twitter-Phänomen})
                \item Social Networking (Social Media) rund um das Barwesen
            \end{itemize}
        \end{block}
    }

    \frame{
        \frametitle{Motivation (1/2)}

        \begin{block}{Ziele von Bars}
            \\~\\
            Gewinnmaximimierung durch
        \end{block}
        \begin{itemize}
            \setlength{\itemsep}{0.3cm}
            \item Popularitätssteigerung
            \item Marketing / Advertisement
            \item Ressourcenplanung
            \item Absetzen von der Konkurrenz
            \item Kommunikation mit dem Kunden
            \item \ldots
        \end{itemize}
    }

    \frame{
        \frametitle{Motivation (2/2)}

        \begin{block}{Ziele der Endkunden}
            \\~\\
            Best mögliches Bar-Erlebnis durch:
        \end{block}
        \begin{itemize}
            \setlength{\itemsep}{0.3cm}
            \item Gute Drinks
            \item Neues erleben
            \item Social Media
            \begin{itemize}
                \setlength{\itemsep}{0.1cm}
                \item[$\rightarrow$] Erlebtes mitteilen / erhalten
                \item[$\rightarrow$] Selbstdarstellung
            \end{itemize}
            \item Kommunikation mit der Bar
            \item \ldots
        \end{itemize}
    }

    \frame{
        \frametitle{Mögliche Datengrundlage (1/2)}

        Bars
        \begin{itemize}
            \item Rezepte und Zutaten als Zusammensetzung von Getränken
            \item Bewertung
            \item Bestellte Getränke
            \item Höhe der Trinkgelder
            \item Stammkundschaft
            \item Preise
            \item \ldots
        \end{itemize}
    }

    \frame{
        \frametitle{Mögliche Datengrundlage (2/2)}

        Endkunden (Bar-Besucher)
        \begin{itemize}
            \item Social Media Profil
            \item Geolocation
            \item Trinkverhalten
            \item Bewertungen
            \item Kommentare
            \item \ldots
        \end{itemize}
    }

    \frame{
        \frametitle{Grundgedanke}
        Der Fokus liegt auf der Erstellung eines Konzepts für eine spätere Datenanalyse.\\~\\
        Durch die Menge an Daten aus verschiedensten Bereichen des Barwesens, sollte eine geeignete Fragestellung
        für eine Bachelorarbeit gefunden werden können.
        \begin{block}{Mögliche Fragestellungen}
            \begin{itemize}
                \item Wie muss eine Datengrundlage beschaffen sein, um eine festgelegte Fragestellung zu beantworten?
                \item Ethik über das gegenseitige Wissen über Bars und Endkunden
                \item Mögliche Korrelationen zwischen Information über Endkunden und Bars
                \item \ldots
            \end{itemize}
        \end{block}
    }

\end{document}