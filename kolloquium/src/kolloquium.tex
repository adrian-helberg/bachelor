%! Author = Adrian Helberg

% Preamble
\documentclass[12pt]{beamer}
% Theme
\usetheme{Boadilla}

% Packages
\usepackage{amsmath}
\usepackage[utf8]{inputenc}
\usepackage{ngerman}
\usepackage{graphicx}
\usepackage{color}
% Remove frame break numbering
\setbeamertemplate{frametitle continuation}{}

\subtitle{Template-basierte Synthese von\\Verzweigungsstrukturen mittels L-Systemen}
\title{Bachelorarbeit Kolloquium}
\author{Adrian Helberg}
\institute{HAW Hamburg}
\date{\today}
\titlegraphic{\includegraphics[scale=0.25]{../images/HAW_Marke_CMYK.pdf}}

% Document
\begin{document}
    \frame{\titlepage}
    \frame{\frametitle{Agenda} \tableofcontents}

    \section{Einleitung}
    \label{sec:thema}
    \begin{frame}[allowframebreaks]
        \frametitle{Einleitung}

        \begin{block}{Titel}
            \color{olive}\underline{\color{black}Template-basierte} \color{teal}\underline{\color{black}Synthese} 
            \color{black}von \color{orange}\underline{\color{black}Verzweigungsstrukturen} \color{black}mittels 
            \color{cyan}\underline{\color{black}L-Systemen}
        \end{block}

        \begin{itemize}
            \item[\color{olive}$\rightarrow$] \color{black}Verschiedene Muster als kleinste zu organisierende Einheit
            \item[\color{teal}$\rightarrow$] \color{black}Verknüpfung von Verzweigungen zu einer neuen Struktur
            \item[\color{orange}$\rightarrow$] \color{black}Baumstrukturen als Ergebnis der Synthese
            \item[\color{cyan}$\rightarrow$] \color{black}Formale Grammatik zur Kodifizierung von Strukturen
        \end{itemize}
    \end{frame}

    \section{Forschungsgegenstand}
    \label{sec:forschungsgegenstand}
    \begin{frame}
        \frametitle{Forschungsgegenstand}

        \begin{itemize}
            \item Fachbegriffe erklären
            \item Theorien zeigen
            \item Diagramme anderer Forschungen zeigen
        \end{itemize}
    \end{frame}

    \section{Methodik}
    \label{sec:methodik}
    \begin{frame}
        \frametitle{Methodik}

        \begin{itemize}
            \item Strukturieren
            \item Datenaufbereitung
            \item Inferieren
            \item Komprimieren
            \item Generalisieren
        \end{itemize}
        \\~\\
        \begin{itemize}
            \item Visualisieren
            \item Randomisieren
        \end{itemize}
    \end{frame}

    \section{Ergebnisse}
    \label{sec:ergebnisse}
    \begin{frame}
        \frametitle{Ergebnisse}
    \end{frame}

    \section{Fazit}
    \label{sec:fazit}
    \begin{frame}
        \frametitle{Fazit}
    \end{frame}

    \section{Items}
    \label{sec:items}
    \begin{frame}
        \frametitle{Items}

        \begin{itemize}
            \item<1-> Test 1
            \item<2-> Test 2
        \end{itemize}
    \end{frame}

    \section{Blocks}
    \label{sec:blocks}
    \begin{frame}
        \frametitle{Blocks}

        \begin{block}{Remark}
            Content
        \end{block}

        \begin{alertblock}{Alert}
            Content
        \end{alertblock}

        \begin{examples}
            Content
        \end{examples}
    \end{frame}

    \section{Columns}
    \label{sec:columns}
    \begin{frame}
        \frametitle{Columns}

        \begin{columns}
            \column{0.5\textwidth}
            Column 1
            \column{0.5\textwidth}
            Column 2
        \end{columns}

    \end{frame}

\end{document}