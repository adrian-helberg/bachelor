% !TEX root = ./manual.tex
%%
% Maintainer
% @author Thomas Lehmann
%
\section{Bekannte Probleme}\label{sec:knownissues}\index{Bugs}
Derzeit sind folgende Problem bekannt, zu denen nur Work Arounds existieren.

\subsection{Glossar wird nicht erstellt}
Das Kapitel mit den Glossareinträgen bleibt leer.

Lösung: Hier gibt es zwei mögliche Ursachen. Zum einen wird das Kommando makeglossarie von der verwendeten IDE nicht aufgerufen. Dann muss dieser Schritt in der Shell über die Kommandozeileneingabe ausgeführt werden. Die zweite Ursache kann sein, dass makeglossaries Perl benötigt, welches evtl. nicht automatisch installiert wurde. Das Fehlen von Perl macht sich beim Aufruf von makeglossaries als explizite Fehlermeldung bemerkbar.

\subsection{Font-Größe auf dem Deckblatt}
Die Font-Größe auf dem Deckblatt passt sich automatisch an. Dabei wird eine Heuristik eingesetzt. Somit kann es passieren, dass die Font-Größe nicht optimal ist.

Lösung: Die Font-Größe muss direkt in der Datei coverpage.tex gesetzt werden.

\subsection{Lorem Ipsum}
Bei einigen \LaTeX -Distributionen fehlt das Paket lipsum, welches Beispieltexte für das Beispiel liefert. Aufgetreten auf  Ubunto 18 Systemen.

Lösung: Das Paket aus dem Style-File durch löschen der Zeile herausnehmen. Das Beispiel kann dann nicht verwendet werden. Auf das Template hat es keinen Einfluss.

\subsection{Glossar fehlt}
Wird kein Glossar erzeugt, so kann es daran liegen, dass der zusätzliche Compile-Schritt nicht ausgeführt wurde. Insbesondere bei Verwendung einer IDE wird dieser nicht standardmäßig ausgeführt.
