\documentclass[
  a4paper,            % DIN A4
  DIV=10,             % Schriftgröße und Satzspiegel
  oneside,            % einseitiger Druck
  BCOR=5mm,           % Bindungskorrektur
  parskip=half,       % Halber Abstand zwischen Absätzen
  numbers=noenddot,   % Kein Punkt hinter Kapitelnummern
  bibtotoc,           % Literaturverzeichnis im Inhaltsverzeichnis
  listof=totoc        % Abbildungs- und Tabellenverzeichnis im Inhaltsverzeichnis
]{scrreprt}
\usepackage{../style/thesisstyle}

%\usepackage{layout}       % Layout Debugging
%\usepackage{showframe}    % Layout Debugging
\usepackage{lipsum}       % for example only
\usepackage{blindtext}    % for example only

\usepackage{float}
\usepackage{pifont}
\usepackage{listings}
\usepackage{color}
\usepackage{caption}
\usepackage{bbding}

\newcounter{nalg}[chapter] % defines algorithm counter for chapter-level
\renewcommand{\thenalg}{\thechapter \arabic{nalg}} %defines appearance of the algorithm counter
\DeclareCaptionLabelFormat{algocaption}{Algorithmus \thenalg} % defines a new caption label as Algorithm x.y

\lstnewenvironment{algorithm}[1][] %defines the algorithm listing environment
{
  \refstepcounter{nalg} %increments algorithm number
  \captionsetup{labelformat=algocaption,labelsep=colon} %defines the caption setup for: it ises label format as the declared caption label above and makes label and caption text to be separated by a ':'
  \lstset{ %this is the stype
    mathescape=true,
    frame=tB,
    numbers=left,
    numberstyle=\tiny,
    basicstyle=\scriptsize,
    keywordstyle=\color{black}\bfseries\em,
    keywords={,input, output, return, datatype, function, in, if, else, foreach, while, begin, end, } %add the keywords you want, or load a language as Rubens explains in his comment above.
    numbers=left,
    xleftmargin=.04\textwidth,
    #1 % this is to add specific settings to an usage of this environment (for instnce, the caption and referable label)
  }
  \lstset{literate=%
      {Ö}{{\"O}}1
      {Ä}{{\"A}}1
      {Ü}{{\"U}}1
      {ß}{{\ss}}1
      {ü}{{\"u}}1
      {ä}{{\"a}}1
      {ö}{{\"o}}1
  }
}
{}

\makeglossaries           % create all glossary entries (remember: run makeglossaries manually)
\loadglsentries{thesisglossaries.tex}  % load acronym, symbol and glossarie entries

\begin{document}
% !TEX root = ../thesis.tex
%
% configurations
%

% text field
%-> replace supervisor names with correct ones
\firstSupervisor{Prof. Dr. Philipp Jenke}
\secondSupervisor{Prof. Dr. Michael Neitzke}

% text field
%-> replace title with your thesis title
\thesisTitle{Template-basierte Synthese von Verzweigungsstrukturen mittels L-Systemen}
\thesisTitleEN{Template-based synthesis of branching structures using L-Systems}

% text field
%-> replace the key words with your own key words
\keywordsDE{Verzweigungsstruktur, 2D Generierung, L-System, Template, Prozedurale Modellierung, Inverse Prozedurale Modellierung, Baumstruktur, Formale Grammatik, Ähnlichkeit}
\keywordsEN{Branching Structure, 2D Generation, L-System, Template, Procedural Modeling, Inverse Procedural Modeling, Tree Structure, Formal Grammar, Similarity}

% text field
%-> replace the text with a description of the thesis
\abstractDE{
Prozedurale Modellierung beschreibt effiziente Methoden zur Erzeugung einer Vielzahl
an Modellen nach bestimmten Regeln. Die Erstellung eines Systems zur Umsetzung solcher Methoden ist durch
einen Mangel an Kontrolle und einer geringen Vorhersagbarkeit der Ergebnisse erschwert.
Diese Bachelorarbeit präsentiert ein System zur Synthese von Verzweigungsstrukturen, die einer benutzerdefinierten
Strukur ähnlich sind. Dabei wird gezeigt, wie sich aktuelle Ansätze der Forschung in einem Programm umsetzen lassen.
Templates werden eingelesen und vom Benutzer zu einer Basisstruktur organisiert.
Anschließend wird ein L-System über die Topologie dieser Struktur inferiert, komprimiert und dann generalisiert.
Nach der Interpretation des L-Systems können vom Benutzer gesetzte Transformationsparameter aus einer Häufigkeitsverteilung
angewendet werden. Zum Schluss wird die resultierende Verzweigungsstruktur visualisiert.
}
\abstractEN{
Procedural modeling describes efficient methods for creating various models according to certain rules.
The creation of a system to implement such methods is complicated due to a lack of control and a low predictability of the results.
This bachelor thesis presents a system for synthesizing branching structures that are similar to custom structures created by a user.
It is shown how current research approaches can be implemented.
A basic structure is build from templates by the user.
Subsequently an L-System is inferred from the topology of this structure, compressed and then generalized.
User-defined transformation parameters from a frequency distribution can be applied to the interpretation of this L-System.
Finally, the resulting branching structure is visualized.
}

% text field
%-> replace jon with your name
\thesisAuthor{Adrian Helberg}

% text field
%-> enter the submission date
\submissionDate{\today}

% switch - uncomment only one
%-> uncomment NDA or public
%\NDA{yes}
\NDA{no}

%-> uncomment cover or cover Corporate Design 2017
\Cover{CD2017}
%\Cover{CD2017NoLogo}
%\Cover{Std2018}

% switch - uncomment only one
%-> uncomment to show list of figures or not
\ListOfFigures{yes}
%\ListOfFigures{no}

% switch - uncomment only one
%-> uncomment to show list of tables or not
\ListOfTables{yes}
%\ListOfTables{no}

% switch - uncomment only one
%-> uncomment to show list of accronyms or not
\ListOfAccronyms{yes}
%\ListOfAccronyms{no}

% switch - uncomment only one
%-> uncomment to show list of symbols or not
\ListOfSymbols{yes}
%\ListOfSymbols{no}

% switch - uncomment only one
%-> uncomment to show list of glossary entries or not
\Glossary{yes}
%\Glossary{no}

% switch - uncomment only one
%-> uncomment the study course you are in
%\studycourse{ITS}
%\studycourse{TI}
\studycourse{AI}
%\studycourse{WI}
%\studycourse{EI}
%\studycourse{REE}
%\studycourse{BMT}
%\studycourse{MAI}
%\studycourse{MIK}
%\studycourse{MA}
    % load all settings

%\layout{}                 % Layout Debugging

\hyphenation{Ba-che-lor-the-sis Mas-ter-the-sis}

% Cover page here, no page number
\ICoverPage

% PDF Metadata
\input{../style/metadata}

% Titlepage is page one even if the number is not shown.
\pagenumbering{roman}
% Title page here
\input{../style/titlepage}

% Abstract page here
\input{../style/abstractpage}

% Table of contents here
\tableofcontents

% List of figures here
\IListOfFigures

% List of tables here
\IListOfTables

% List of accronyms here
\IListOfAccronyms

% List of symbols here
\IListOfSymbols

% Uncomment if list of source code is needed (rarely).
\lstlistoflistings  % requires package listings, needs to uncommenting of usepackage

% path to the chapters folder is set to find the images used there
\graphicspath{ {./chapters/} }

% Chapters
\clearpage
\pagenumbering{arabic}
% @author Arian Helberg

\chapter{Einleitung}
Effizientes Objektdesign und -modellierung sind entscheidende Kernfunktionalitäten in verschiedenen Bereichen der
digitalen Welt.
Da die Erstellung geometrischer Objekte für Laien unintuitiv ist und ein großes Maß an Erfahrung und Expertise
erfordert, ist dieses stetig komplexer werdende Feld für Neueinsteiger nur sehr schwer zu erschließen.
Die Forschung liefert hierzu einige Arbeiten zur prozeduralen Modellierung, um digitale Inhalte schneller
und automatisiert zu erstellen.
Gerade wenn es um die Darstellung natürlicher Umgebung geht, ist die Erstellung von ähnlichen Objekten, wie
zum Beispiel verschiedene Bäume derselben Gattung eines Waldes, ein schwieriges Problem.
Kleine Änderungen in prozeduralen Systemen können zu großen Veränderungen der Ergebnisse führen.
Darum beschäftigt sich die inverse prozedurale Modellierung unter anderem mit dem Inferieren von Regeln
und Mustern aus gegebenen Objekten, um diese nach bestimmten Regeln zu modellieren.
Ein wichtiges Werkzeug hierbei ist die Verwendung formaler Grammatiken als fundamentale Datenstruktur
der Informatik, um Strukturen zu beschreiben.
Eine spezielle Untergruppe sind die L-Systeme, die häufig bei der Beschreibung
von Verzweigungsstrukturen und Selbstähnlichkeit zum Einsatz kommen.
\\~\\
Diese Arbeit beschäftigt sich mit der Erstellung eines prozeduralen Systems zur Synthese von Ähnlichkeitsabbildern.
Hierzu soll über eine Benutzerschnittstelle eine Struktur erzeugt werden, aus der ein parametrisiertes L-Systems
inferiert werden kann, welches dann zur Generierung von ähnlichen Strukturen genutzt werden kann.

\newpage

\section{Problemstellung}
Während Methodiken zur Kodifizierung bestimmter Strukturen in den Bereich der prozeduralen Modellierung fällt,
findet das Ableiten von Regeln Anwendung in der inversen prozeduralen Modellierung.
Mit dem digitalen und naturwissenschaftlichen Fortschritt steigt die Anwendung immer komplexerer Strukturen, die
ein Herausarbeiten der schwer zu kontrollierenden, prozeduralen Regeln immer schwieriger machen.
\\~\\
Ein Beispiel hierzu aus der Gaming-Industrie ist die frühere Verwendung unorganisierter Modelle (ugs. Dreieckssuppe).
Diese Modelle weisen keine hierarchische Struktur auf und werden erst durch Datenstrukturen, wie bspw. Octrees organisiert.
Unorganisierte Modelle sind nur bedingt wiederverwendbar, da nur die vorliegende Modellierung genutzt werden kann.
Es besteht eine hohe Speicherkomplexität bei geringer Laufzeitkomplexität.
Für kleinste Veränderungen am Modell ist eine erneute Modellierung erforderlich, die wiederum Speicher benötigt, um sie
persistent speichern zu können.
Heutzutage werden die Objekte nach bestimmten Kriterien organisiert, um eine automatisierte Modellierung durch Algorithmen
zu ermöglichen, und so aus einer Grundstruktur weitere Modelle zu erzeugen.
Speicher- und Laufzeitkomplexität nähern sich an.
Aus diesem Grund werden allgemeingültige, vielseitig anwendbare Algorithmen gesucht, die bestimmte natürliche Eigenschaften
von Strukturen herausarbeiten (Reverse Engineering), um diese für die (inverse) prozedurale Modellierung zur
Verfügung zu stellen.
\\~\\
Diese Arbeit soll zeigen, wie sich durch die Erstellung eines Systems zur Generierung von ähnlichen Strukturen aus einer
Basistruktur, aktuelle Ansätze aus der Forschung in einem Programm umsetzen lassen.

\newpage

\section{Ziele}
Die Erstellung eines Systems zur Synthese von ähnlichen Strukturen aus einer Basis-struktur soll zentrale Aufgabe dieser
Arbeit sein.
Zunächst wird ein System benötigt, das eine Verzweigungsstruktur bestimmt, aus der die ählichen Strukturen erzeugt
werden können.
Um bestimmte Methodiken und Algorithmen der aktuellen Forschung auf die Eingabestruktur anwenden zu können, sollte diese
in einer Datenstruktur organisiert werden.
Anschließend kann ein L-System aus der Struktur inferiert werden.
Die inferierte Grammatik kann durch diverse Algorithmen verändert werden, bis das erzeugte L-System in der Lage ist
Ähnlichkeitsstrukturen zu bilden.

\section{Methodik}
Der Benutzer des Systems legt atomare Strukturen in Form von Zeichenketten an, die vom Programm eingelesen und
als Templates zur Verfügung gestellt werden.
Die Templates sind beliebig und können eine einfache Linie oder eine komplexe Verzweigung darstellen.
Werden diese Strukturen auf der graphischen Oberfläche platziert und mit diversen Transformationen verändert, spricht man
von Instanzen oder Template-Instanzen.

\subsection*{Strukturieren}
Der Benutzer verwendet die grafische Schnittstelle, um aus einzelnen Templates eine zusammenhängende Basisstruktur zu
erstellen.
Neben der Position der Instanzen werden Transformationsparameter, wie Rotation oder Skalierung, angepasst.

\subsection*{Visualisieren}
Der aktuelle Stand der Strukturierung ist jederzeit sichtbar.
Liniensegmente und Bindungselemente werden in einem graphischen Element visualisiert und für eine Interaktion zur Verfügung
gestellt.

\subsection*{Datenaufbereitung}
Das Ergebnis der Strukturierung wird in einer Baumstruktur
organisiert, in der jeder Knoten einer bestimmten Template-Instanz entspricht und die eingehende Kante die
geometrischen Transformationen relativ zum Elternknoten beschreibt.

\subsection*{Inferieren}
Aus der Baumstruktur kann eine formale Grammatik in Form eines L-Systems abgeleitet werden.
Diese Grammatik beschreibt lediglich die erstellte Basisstruktur und beinhaltet keine Transformationsparameter, da hier
nur auf die Topologie des Baumes und nicht auf geometrische Unterschiede der Instanzen eingegangen wird.

\subsection*{Komprimieren}
Um sich wiederholende Produktionsregeln zu vermeiden und so sowohl das Alphabet, als auch die Produktionsregelmenge
zu komprimieren, wird die Baumstruktur nach identischen, maximalen Unterbäumen durchsucht und durch zusammengefasste
Instanzen vereinfacht.
Hierbei gilt die Baumstruktur selbst nicht als Unterbaum.

\subsection*{Generalisieren}
Ähnliche Produktionsregeln des L-Systems werden mithilfe einer Kostenfunktion zusammengefasst, um diese mit
nicht-deterministischen Regeln zu generalisieren.
Die Kostenfunktion entscheidet hierbei über den Grad der Ähnlichkeit zweier Produktionsregeln.

\subsection*{Randomisieren}
Jedes Symbol der Grammatik nimmt eine Liste an Parametern entgegen, die nach bestimmten Kriterien pseudo-randomisiert
werden, um Variationen von Template-Instanzen zu erstellen.
Das Ausführen des L-Systems kann nun Ähnlichkeitsstrukturen für die Basisstruktur erzeugen.

\section{Aufbau}
Die Methodik zum Umsetzen des beschriebenen Systems wird wie folgt umgesetzt.

\begin{figure}[H]
    \centering
    \includegraphics[width=14cm]{../images/System.PNG}
    \caption[Systemarchitektur]{Architektur des Systems mit einigen Datenstrukturen}
\end{figure}

Die Darstellung zeigt eine grobe Übersicht der Anwendungsfälle innerhalb des Systems.
Der Benutzer strukturiert vom System importierte Templates zu einer Basisstruktur mittels grafischer Bedienelemente.
Die Template-Instanzen werden hierbei in einer Baumtopologie organisiert.
Anschließend wird ein L-System aus der Baumstruktur generiert und komprimiert.
Das kompakte L-System wird generalisiert und ausgeführt.
Zuletzt wird das abgeleitete Axiom parametrisiert und sichtbar gemacht.
% @author Arian Helberg

\chapter{Grundlagen}

Die Modellierung mithilfe von Grafiksoftware ist eine vergleichbar händische, langwierige Erstellung von
Objekten.
Hierbei hat der Designer (= Modellierer) die volle Kontrolle über die Strukturen des Objektes.\\
Bei der prozeduralen Modellierung werden spezifische Strukturen eines zu erstellenden physikalischen Objektes
generalisiert und meist über eine Grammatik und globale Parameter abgebildet.
Während bei der klassischen Modellierung die menschliche Intuition und bei der prozeduralen Modellierung eine
parametrisierte Grammatik vorausgesetzt wird, arbeitet IPM mit bestehenden Modellen und extrahiert ("`lernt"')
die Strukturen des Objektes, die automatisch in eine formale Grammatik überführt werden können.
Die Generierung von prozeduralen Modellen ist ein wichtiges, offenes Problem.
\\~\\
Aktuelle Ansätze sind:
\begin{itemize}
    \item Segmentierung von geometrischen Objekten in Ähnlichkeitsgruppen, um Muster (\textit{engl. Patterns}) zu
    erkennen
    \item Kontrollierte Generierung durch Finden optimaler Prameter und Regeln
\end{itemize}

\section{Grundbegriffe}

\begin{itemize}
    \item \textbf{Modellierung}: Um einen physikalischen Körper in ein digitales Objekt zu überführen, wird mithilfe
    von Abstraktion (oder Modellierung) ein mathematisches Modell erstellt, das diesen Körper formal beschreibt.
    3D Grafiksoftware, wie Blender~\cite{blender}, wird genutzt um geometrische Körper zu modellieren, texturieren
    und zu animieren.
    \item \textbf{Prozedurale Modellierung}: \textit{"`It encompasses a wide variety of generative techniques that
    can (semi-−)automatically produce a specific type of content based on a set of input
    parameters."'} \\
    "`Prozedurale Modellierung beschreibt generative Techniken, die \\(semi-)automatisch spezifische, digitale
    Inhalte anhand von deskriptiven Parametern erzeugen"' (\textit{Übersetzt durch den Autor})
    \item \textbf{Inverse prozedurale Modellierung (IPM)}:\textit{Aliaga et al.}
    spricht bei der inversen prozeduralen Modellierung von dem Finden einer prozeduralen Repräsentation von
    Strukturen bestehender Modelle.
\end{itemize}

\section{Grundlegende Arbeiten}

Grundlagen
\begin{itemize}
    \item prozedurale Modellierung~\cite{smelik_2014}
    \item inverse prozedurale Modellierung~\cite{aliaga_2016}
    \item L-Systeme~\cite{lindemayer_1968}
    \item parametrisiert L-Systeme~\cite{prusinkiewicz_1993}
    \item Logo-Turtle~\cite{prusinkiewicz_1986}
    \item inverse Generierung von L-Systemem~\cite{higuera_2010}
\end{itemize}

"`Not much work addresses inverse procedural modeling of branching structures"'\\~\\

Modellieren von
\begin{itemize}
    \item Bäumen und Landschaften~\cite{deussen_2010}
    \item Fassaden~\cite{alhalawani_2013}
    \item Gebäuden~\cite{mueller_2006}
    \item Städten~\cite{parish_2001}
\end{itemize}

\subsection{Softwaretechnik}

Eine Fallstudie der Universität Karlsruhe~\cite{muller_2001} untersucht den Einsatz der Softwaretechnik Extreme
Programming(XP) im Kontext der Erstellung von Abschlussarbeiten im Universitätsumfeld.
Hierzu werden folgende Schlüsselpraktiken untersucht:
\begin{itemize}
    \item XP als Softwaretechnik zur schrittweisen Annäherung an die Anforderungen eines Systems
    \item Änderung der Anforderungen an das Systems
    \item Funktionalitäten (\textbf{Features}) werden als Tätigkeiten des Benutzers (\textbf{User Stories}) definiert
    \item Zuerst werden Komponententests (Modultests) geschrieben und anschließend die Features (Test-driven Design)
    \item Keine seperaten Testing-Phasen
    \item Keine formalen Reviews oder Inspektionen
    \item Regelmäßige Integration von Änderungen
    \item Gemeinsame Implementierung (Pair Programming) in Zweiergruppen
\end{itemize}
Aus der Fallstudie geht hervor, dass Extreme Programming einige Vorteile bei der Bearbeitung eines Softwareprojektes
einer Bachelorarbeit bietet.
Zum einen können sich Anforderungen an das zu erstellende System durch parallele Literaturrecherche ändern, zum
anderen können Arbeitspakete durch Releases abgebildet werden.

\subsubsection{L-Systeme}

Lindenmayer~\cite{lindemayer_1968} führt eine mathematische Beschreibung zum Wachstum fadenförmiger Organismen ein.
Sie zeigt, wie sich der Status von Zellen infolge ein oder mehrerer Einflüsse verhält.
L-Systeme sind Ersetzungssysteme, die zur formalen Beschreibung von Zellteilung eingeführt werden, und atomare Teile
mithilfe von Produktionsregeln ersetzen.
Weiter werden Symbole zur formalen Beschreibung von Verzweigungen, die von Filamenten abgehen, genutzt.
Die bekanntesten L-Systeme sind zeichenketten-basiert und werden von \textit{Noam Chomsky} eingeführt.
Sie ersetzen parallel Buchstaben eines Wortes, die von einer Grammatik über eine Sprache akzeptiert werden.
L-Systeme können unter anderem parametrisiert oder nicht-parametrisiert und kontextfrei oder kontextsensitiv sein.
\\~\\
\underline{Formalismen}\\~\\
Ein L-System ist ein Tupel und hat folgende Form:
\begin{center}
    $\mathcal{L}=\langle M,\omega,R \rangle$, mit
\end{center}
\begin{itemize}
    \item $M$ als Alphabet, das alle Symbole enthält, die in der Grammatik vorkommen können,
    \item $\omega$ als Axiom oder "`Startwort"' und
    \item $R$ als Menge aller Produktionsregeln, die für $\mathcal{L}$ gelten
\end{itemize}
\\~\\
Das Alphabet eines parametrischen Systems enthält Module (Symbole mit Parametern) anstatt Symbole:
\begin{center}
    $M=\{A(P),B(P),\dots\}$ mit
\end{center}
\begin{itemize}
    \item $P=p_1,p_2\dots$ als Modulparameter
\end{itemize}
Zeichen des Alphabets, die Ziel einer Produktionsregel sind, heißen Variablen.
Alle anderen Zeichen aus $M$ sind die Konstanten.
\\~\\
Das Axiom $\omega$ ist eine nicht-leere Sequenz an Modulen aus $M^+$ mit
\begin{itemize}
    \item $M^+$ als Menge aller möglichen Zeichenketten aus Modulen aus $M$
\end{itemize}
\\~\\
Produktionsregeln sind geordnete Paare aus Wörtern über dem Alphabet, die bestimmte Ersetzungsregeln umsetzen.
Hierbei werden Symbole aus einem Wort, die einer rechten Seite (\textit{engl. right hand side (RHS)}) einer
Produktionsregel entsprechen, durch die linke Seite des Paares (\textit{engl. left hand side (LHS)}) ersetzt.
Sie sind folgendermaßen aufgabaut:
\begin{center}
    $A(P)\rightarrow x,x\in M^*$
\end{center}
$M^*$ ist die Menge aller möglichen Zeichenketten von M inklusive der leeren Zeichenkette $\varepsilon$.
Ist die RHS jeder Produktionsregel ein einzelnes Symbol und gibt es zu jeder Variablen eine Regel, spricht man
von einem kontextfreien, andernfalls von einem kontextsensitiven L-System.

\subsubsection{Logo-Turtle-Algorithmus}

Der Logo-Turtle-Algorithmus~\cite{prusinkiewicz_1986} beschreibt ein Vorgehen zur graphischen Beschreibung von L-Systemen, bei dem
jeder Buchstabe in einem Wort einer bestimmten Zeichenoperation zugewiesen wird.
So kann aus einem L-System ein grafisches Muster generiert werden, das mit einer Abfolge von Zeichenbefehlen an
eine "`Schildkröte"' gezeichnet wird.
Das Triplett $(x,y,\theta)$ definiert den Status (\textbf{State}) der Schildkröte.
Dieser setzt sich aus der aktuellen Position $\left(\begin{smallmatrix} x \\ y \end{smallmatrix}\right)$ und dem
aktuellen Rotationswinkel $\theta$, der die Blickrichtung bestimmt, zusammen.\\~\\
Der Algorithmus kann als Komprimierung eines geometrischen Musters gesehen werden.
Folgende Symbole mit zugehörigen Steuerungsbefehlen und Statusveränderung sind definiert:
\begin{center}
    \begin{tabular}{lll}
        % ROW 1
        \textbf{Symbol} & \textbf{Steuerung} & \textbf{Statusveränderung} \\
        \hline \\
        % ROW 2
        $F(d)$ &
        \begin{minipage}{0.6\textwidth}
            Gehe vom derzeitigen Punkt $p_1$ $d$ Einheiten in die Blickrichtung zu dem Punkt $p_2$.
            Zeichne ein Liniensegment zwischen $p_1$ und $p_2$
        \end{minipage} &
        \begin{minipage}{0.4\textwidth}
            ja, neue Position $p_2$
        \end{minipage} \\
        \\ \hline \\
        % ROW 3
        $+(\alpha)$ &
        \begin{minipage}{0.6\textwidth}
            Setzt neuen Rotationswinkel $\theta=\theta+\alpha$
        \end{minipage} &
        \begin{minipage}{0.4\textwidth}
            ja, neuer Rotationswinkel $\theta$
        \end{minipage} \\
        \\ \hline \\
        % ROW 4
        $-(\alpha)$ &
        \begin{minipage}{0.6\textwidth}
            Setzt neuen Rotationswinkel $\theta=\theta-\alpha$
        \end{minipage} &
        \begin{minipage}{0.4\textwidth}
            ja, neuer Rotationswinkel $\theta$
        \end{minipage} \\
        \\ \hline \\
        % ROW 5
        $[$ &
        \begin{minipage}{0.6\textwidth}
            Lege den aktuellen State auf einen Stack
        \end{minipage} &
        \begin{minipage}{0.4\textwidth}
            nein
        \end{minipage} \\
        \\ \hline \\
        % ROW 6
        $]$ &
        \begin{minipage}{0.6\textwidth}
            Hole den State vom Stack und überschreibe den aktuellen mit diesem
        \end{minipage} &
        \begin{minipage}{0.4\textwidth}
            nein
        \end{minipage}
    \end{tabular}
\end{center}
\newpage
Alles zwischen den Symbolen $[$ und $]$ wird als Verzweigung interpretiert.
\begin{center}
    Bsp. $FF[FF]F$ mit Verzweigung $[FF]$
\end{center}

\section{Verwandte Arbeiten}

Bei \textit{Inverse Procedural Modeling of Branching Structures by Inferring L-Systems}\cite{guo_2020} geht es um ein
Modell zum Lernen von L-Systemen von Verzweigungsstrukturen mithilfe maschinellen Lernens (\textit{Deep
Learning}) anhand beliebiger Grafiken.
Hierzu werden atomare Strukturen mit einem neuronalen Netz erkannt, eine hierarchische Topologie (Baumstruktur)
aufgebaut, aus der ein L-System inferiert und mit einem \textbf{Greedy} Algorithmus optimiert wird.
Ausgabe des Systems ist ein generalisiertes L-System, aus dem ähnliche Strukturen, wie die der Inputgrafik,
erstellt werden können.\\
Aus dieser Quelle werden folgende Konzepte genutzt:
\begin{itemize}
    \item Nutzer einer Baumstruktur zur Organisation von genutzten atomaren Verzweigungsstrukturen
    (\textbf{Templates}) mit Knoten für Templates und Kanten für geometrische Transformationen
    \item Untersuchen der Baumstruktur auf Wiederholungen
    \item Parametrisierte L-Systeme (L-System mit \textbf{Modulen}) zur Abbildung von Transformationsparametern
    \item Kostenfunktion zur Bewertung eines L-Systems
\end{itemize}

Inverse prozedurale Modellierung anhand von Hausfassaden~\cite{martinovic_2013}, 2D-Anordnungen
~\cite{ellis_2018,stava_2010}, biologischen Bäumen~\cite{stava_2014} und urbanen Strukturen~\cite{nishida_2016}
\\~\\
Polynomiale Algorithmen zum Inferrieren von L-Systemen~\cite{mcquillan_2018}
\\~\\
Framework zum Inferrieren von L-Systemen aus Vektordaten~\cite{stava_2010}
\\~\\
Generalisieren einen Regelsets aus einem Inputset mit Markov Chain Monte Carlo~\cite{talton_2012, talton_2011}
\\~\\
Lernen von Regeln zum Layout für Gebäude~\cite{martinovic_2013}
\\~\\
Inverse prozedurale Modellierung von Fassaden~\cite{xiao_2008}
% @author Arian Helberg

\chapter{Konzepte}
Mit der Erstellung eines Programms zur Synthetisierung von Ähnlichkeitsabbildungen einer vom Benutzer
erstellten Verzweigungsstruktur soll die Praktibilität aktueller Forschungsansätze untersucht werden.
\\~\\
Folgende Kernkonzepte werden erläutert und umgesetzt:
\begin{itemize}
    \item Visualisierung und prozessorientiertes Erstellen von Basisstrukturen,
    \item Organisation in einer prozessoptimierten, baumähnlichen Topologie,
    \item L-System-Repräsentationen,
    \item Algorithmen zur Inferierung, Komprimierung, Generalisierung und
    \item Verarbeitung von Transformationsparametern
\end{itemize}

\section{Probleme \& Lösungsansätze}
\label{probleme}

\subsection*{Visualisierung}
Um eine geführte Erstellung der Basisstruktur zu ermöglichen, muss diese während der Erstellung sichtbar gemacht werden.
Hierzu werden die Templates in Form von Zeichenketten angelegt und mittels Turtle-Grafik visualisiert.
Eine Turtle-Grafik beschreibt die Interpretation einer Zeichenkette als Bild durch Ausführen eines Logo-Turtle-Algorithmus.
Weiter wird auch zur Evaluation von Ergebnissen eine Visualisierung benötigt.
Da die Verzweigungsstrukturen in L-System-Repräsentation vorliegen, wird hierzu eine Interpretationsfunktion benötigt,
die diese Ersetzungssysteme in Bildform darstellen.
Ein L-System wird durch Ausführung in eine erweiterte Zeichenkette überführt und als Turtle-Grafik beschrieben
~\cite{prusinkiewicz_1986}.

\subsection*{Basisstruktur}
Der Benutzer nutzt grafische Bendienelemente, um eingelesene Templates auszuwählen, Transformationsparameter anzupassen
und um anschließend die Instanzen der Basisstruktur hinzuzufügen.
Im Folgenden wird diese Basisstruktur u.a. Grundstruktur und Eingabestruktur genannt.

\subsection*{Baumstruktur}
Um Grundstrukturen mittels verschiedener Algorithmen untersuchen zu können, werden die einzelnen Template-Instanzen in
einer baumähnlichen Struktur organisiert.
Transformationsparameter einer Instanz beschreiben die räumlichen Veränderung gegenüber des zugrundeliegenden Templates
und haben daher keine Aussagekraft in Bezug auf die Strukturtopologie der Basisstruktur.
Diese Arbeit fokussiert sich auf topologische Eigenschaften von Verzweigungsstrukturen (z.B. Rekursionen).
Darum bilden die einzelnen Template-Instanzen die Knoten der Baumtopologie, während die Kanten die räumlichen Transformationen
darstellen.
So wird eine datenstrukturelle Trennung zwischen Topologie und räumlichen Transformationen geschaffen.
Diese baumähnliche Struktur ist inspiriert durch~\cite{guo_2020}, Kapitel 4.2 \textit{Grammar inference}.

\subsection*{Inferieren}
Das Smallest Grammar Problem, also das Finden der kleinsten, kontextfreien Grammatik, welche eine bestimmte Zeichenkette
generiert, ist ein offenes Problem der Informatik mit einem Annäherungsverhältnis von weniger als $\frac{8569}{8568}$
(NP-hard)\cite{charikar_2005}.
Primär wird in der Forschung nach Algorithmen gesucht, die ein akzeptables Ergebnis liefern.
In dieser Arbeit wird ein Algorithmus präsentiert, der die Knoten der Baumstruktur in einzelne Symbole umwandelt, mit
Produktionsregeln verknüpft und diese dem resultierenden L-System hinzufügt.
Dieses L-System repräsentiert lediglich die Eingabestruktur.

\newpage

\subsection*{Komprimieren}
Um ein kompaktes, gewichtetes L-System zu erzeugen, werden sich wiederholende Unterrbäume gesucht und ersetzt.
Eine Gewichtung wird angewendet, um das zu erzeugende L-System mit kleiner Regelmenge oder mit
großer Regelmenge auszustatten.
Eine Kostenfunktion stellt hierbei die Anzahl Symbole aller RHS der Produktionsregeln mit der Menge an Anwendungen der
LHS gegenüber.

\subsection*{Generalisieren}
Da das kompakte L-System eine Repräsentation der vom Benutzer erzeugten Verzweigungsstruktur darstellt, werden ähnliche
Regeln miteinander verbunden (Merge) und mit einer Wahrscheinlichkeit versehen, um nicht-deterministische Regeln hinzuzufügen.
Eine weitere Kostenfunktion bewertet den Merge zweier Produktionregeln und wendet eine Gewichtung über die Länge der Grammatik
zur Änderungsdistanz der alten (ohne Merge) zur neuen Grammatik um.

\subsection*{Transformationen}
Die vom Benutzer vergebenen Werte der Transformationsparameter werden während der Erstellung der Eingabestruktur in
einer Häufigkeitsverteilung organisiert und bei Ausführung des generalisierten L-Systems angewendet.
So werden Transformationen nach ihrer statistischen Häufigkeit angewendet.

\newpage

\section{Workflows \& Algorithmen}
\label{algo}

\subsubsection*{Verzweigungsstruktur erstellen}
Um als Benutzer des Systems eine Verzweigungsstruktur zu erstellen, wird folgender Arbeitsablauf umgesetzt:
\begin{algorithm}[caption={Erstellen einer Verzweigungsstruktur}, label={alg1}]
    Erster Anker ist vorselektiert
    Wiederhole, bis Struktur fertiggestellt ist:
        Selektiere ein Template aus der Liste
        Setzte Parameter
        Bestätige Auswahl und Parameter
        Zeichne ausgewähltes Template mit Parametern
        Wähle nächsten Anker aus
\end{algorithm}

\subsubsection*{L-System inferieren}
Aus der Verzweigungsstruktur kann nun ein L-System erzeugt werden.
Hierzu wird ein neuer Algorithmus (siehe Algorithmus 32) präsentiert:\\~\\
Zunächst werden L-System-Komponenten initialisiert (Zeile 2-6) für $\mathcal{L}=\langle M,\omega,R \rangle$:
\begin{itemize}
    \item Das Alphabet wird mit den Symbolen F und S initialisiert, da F als Repräsentation einer grundlegenden
    Zeichenoperation und S als Axiom in jeder Anwendung des Algorithmus vorkommt.
    \item Die erste Produktionsregel $\alpha$ umfasst die Abbildung des Axioms auf einen neues Symbol, das nicht im Alphabet
    vorkommt.
    Das Alphabet wird stets um ein Symbol lexikografischer Ordnung ergänzt:
    \begin{itemize}
        \item Bsp.: $\{A,B,C\}$ wird ein neues, unbekanntes Symbol hinzugefügt $\rightarrow \{A,B,C,D\}$
    \end{itemize}
    \item Die Variable $\beta$ hält zu untersuchende Knoten der Baumtopologie, welche nach Breitensuche iteriert werden.
    Die erste Iteration startet bei Wurzelknoten S.
    \item Als letzten Schritt der Initialisierung wird dem Alphabet ein neues Symbol hinzugefügt, das durch die Variable
    $\gamma$ gehalten wird.
\end{itemize}
Die Schleife des Algorithmus beschäftigt sich mit der Iterierung des Baumes und dem Erstellen neuer Symbole und
Produktionsregeln für das resultierende L-System (Zeile 8-17):
\begin{itemize}
    \item Die in den Knoten des Baumes gehaltenen Template-Instanzen entprechen einer Zeichenkette, die durch eine
    Turtle-Grafik interpretiert, einem vom Benutzer transformierten Template entspricht.
    Diese Zeichenkette wird in $\delta$ gespeichert
    \item In Zeile 9-12 wird die genannte Zeichenkette auf Verzweiungsvariablen ($A-Z$; $F$ ausgeschlossen) untersucht.
    Diese werden durch ein neues Symbol, das dem Alphabet hinzugefügt wird, ersetzt. Anschließend wird eine Produktionregel,
    die auf die veränderte Zeichenkette abbilded, der Produktionsregelmenge hinzugefügt.
    \item Zeile 13 prüft, ob es ein Symbol im Alphabet gibt, das nicht als Ziel einer Produktionsregel in der Produktionsregelmenge
    definiert ist. Ist dies der Fall, wird dieses als Ziel der nächsten Produktion gesetzt.
    Andernfalls schließt der Algorithmus ab
    \item Schleifen-Attribut ist hier das Setzen des nächsten Knotens am Ende der Schleife (Zeile 17)
\end{itemize}

\begin{algorithm}[caption={Inferieren eines L-Systems aus einer Baumstruktur}, label={alg2}]
    Initialisierung:
        $M=\{F,S\}$
        $\omega=S$
        $R \gets \{\alpha$: $S \rightarrow A\}$
        $\beta=$ nächster Knoten
        $M \gets \gamma \in \{A,B,\dots,Z\}$, mit $\gamma \notin M$
    Schleife:
        $\delta=$ Wort von $\beta$
        $\forall \{A,B,\dots,Z\} \setminus F \in \delta:$
        Ersetze mit $\zeta \in \{A,B,\dots,Z\}$, mit $\zeta \notin M$
        $M \gets \zeta$
        $R \gets \{\gamma\rightarrow\delta\}$
        Wenn es ein Symbol $\eta$ in $M\setminus\{F,S\}$ gibt mit $\{\eta \rightarrow bel.\} \notin R$:
            $\gamma=\eta$
        Sonst:
            Breche Schleife ab
        $\beta=$ nächster Knoten
\end{algorithm}

\subsubsection*{L-System komprimieren}
\citeauthor{guo_2020} führt einen Algorithmus zum Inferieren einer Grammatik aus einer Baumstruktur ein~\cite{guo_2020}.
Zum Einen wird ein L-System aufgebaut, zum Andern die Baumtopologie durch Finden maximaler Subbäume reduziert.
Die Reduktion wird im folgenden Algorithmus 33 adaptiert:\\~\\
Initialisierung (Zeile 2-5):
\begin{itemize}
    \item Das L-System, welches der Eingabe des Algorithmus entspricht, wird ausgeführt und
    die resultierende Zeichenkette wird in $\mathcal{L^+}$ gespeichert.
    \item Ein Gewichtungsparameter $w_l$ wird eingeführt, der eine Kostenfunktion (Zeile 11) nach Anzahl an Symbolen
    der RHS von Produktionsregeln und Anzahl deren Anwendung gewichten soll.
    \item Anschließend wird ein maximaler Subbaum durch geschachtelte Iteration des Baumes $T$ gesucht und als $T'$ gesetzt.
    \item Es werden ausschließlich maximale Unterbäume behandelt, die mindestens zwei mal im Baum vorkommen.
\end{itemize}
Die Schleife (Zeile 7-13) stellt die Reduzierung dar:
\begin{itemize}
    \item Zunächst werden alle Vorkommen des maximalen Unterbaumes durch ein neues Symbol ersetzt.
    \item Aus diesem Subbaum wird ein L-System inferriert, das wiederum in eine erweiterte Zeichenkette ausgeführt wird.
    \item Die Zeichenkette wird als LHS einer neuen Produktionsregel gesetzt, die auf das neue Symbel abzielt.
    \item Das alte L-System kann nun mit dem veränderten L-System mittels Kostenfunktion verglichen werden (Zeile 9):
    Liegen die Kosten des veränderten L-Systems unter den Kosten des Alten, wird die Reduktion beendet.
    Andernfalls gilt der Subbaum nun als Eingabebaum und das veränderte Ersetzungssystem als Eingabe-L-System.
\end{itemize}

\begin{algorithm}[caption={Erstellen eines kompakten L-Systems mit Gewichtung $w_l$}, label={alg3}]
Initialisierung:
    $\mathcal{L}^+ \leftarrow L_s$
    $\mathcal{L}=\emptyset$
    Setze Gewichtungsparameter $w_l \in [0,1]$
    Finde maximalen Unterbaum $T'$ aus $T$ mit Wiederholungen $n>1$
Schleife (Reduzierung):
    Ersetze alle Vorkommen von $T'$ mit dem selben Symbol $\gamma \in \{A,B,\dots,Z\}$
    $R \leftarrow \{\gamma \rightarrow L_s\}$ mit $L_s$ aus $T'$, $R$ aus $\mathcal{L}$
    Wenn $C_i(\mathcal{L}) \geq C_i(\mathcal{L}^+)$:
        Breche Schleife ab
    $T \leftarrow T'$
    $\mathcal{L}^+ \leftarrow \mathcal{L}$
    Finde maximalen Unterbaum $T'$ aus $T$ mit Wiederholungen $n>1$
\end{algorithm}

\begin{algorithm}[caption={Kostenfunktion $C_i$ mit Gewichtung $w_l$}, label={alg4}]
$C_i(\mathcal{L})= \sum\limits_{A(P) \rightarrow M^* \in \mathcal{L}} w_l * |M^*| + (1 - w_l) * N(A(P)\rightarrow M^*)$
\end{algorithm}
mit $N(\cdot)$ als Zählfunktion für die Anzahl Wiederholungen einer \textit{LHS} einer Regel in einem
ausgeführten L-System.

\subsubsection*{L-System generalisieren}
Da das kompakte L-System eine Repräsentation der vom Benutzer erzeugten Verzweigungsstruktur darstellt, werden
nun ähnliche Regeln miteinander verbunden und mit einer Wahrscheinlichkeit versehen, um nicht-deterministische
Regeln hinzuzufügen.
Sowohl Längenfunktionen, Kostenfunktionen und Distanzalgorithmen, als auch der Grundalgorithmus sind aus einer Arbeit
von~\citeauthor{guo_2020} entnommen und bauen sich wie folgt auf~\cite{guo_2020}:

\begin{itemize}
    \item Die Längenfunktion $L$, die auf eine Grammatik angewendet wird, summiert die Anzahl Symbole des Alphabets mit
    der Anzahl an RHS der Produktionsregeln und misst somit die Gesamtheit aller Symbole, die das L-System abbilden soll
    (Länge der Grammatik).
    \item Der Abstand zweier Zeichenketten kann mit der \textit{String Edit Distance} ermittelt werden. Diese wird über
    die Funktion $D_s$ abgebildet. Hierbei wird die Anzahl an Operationen summiert, die für die Überführung einer Zeichenkette
    in eine Andere nötig sind (Zeichenkettenaustausch, -einschub und -löschung)
    \item Mit $D_s$ kann nun auch der Abstand zweier Grammatiken zueinander bestimmt werden. Diese Funktion wird mit
    $D_g$ abgebildet.
    \item Die Kostenfunktion $C_g$ nutzt die Länge und Distanz von Grammatiken, um Kosten einer Überführung einer Grammatik
    in eine andere messen zu können. Sie berechnet also sie die Kosten, um $L^*$ in $L^+$ zu überführen.
    Der Parameter $w_0$ gewichtet hierbei die Differenz der Länge der Grammatiken und die Anzahl Operationen, die zur
    Überführung nötig sind. Die Überführungskosten werden in der Variable $C^{old}_g$ zur Verfügung gestellt.
\end{itemize}

\begin{algorithm}[caption={Längenfunktion $L$ für Grammatiken}]
$L(\mathcal{L}) = |M| + \sum\limits_{A(P) \rightarrow M^* \in \mathcal{L}} |M^*|$
\end{algorithm}

\begin{algorithm}[caption={Grammar Edit Distance}]
$D_g(\mathcal{L}^+, \mathcal{L}^*)= \sum\limits_{(A(P) \rightarrow M^*_A , B(P) \rightarrow M^*_B) \in M(\mathcal{L^+} \rightarrow \mathcal{L^*})} D_s(M^*_A, M^*_B)$
\end{algorithm}

\begin{algorithm}[caption={Kostenfunktion $C_g$ mit Gewichtung $w_0$}]
$C_g(\mathcal{L}^*, \mathcal{L}^+) = w_0 * (L(\mathcal{L}^*) - L(\mathcal{L}^+)) + (1 - w_0) + D_g(\mathcal{L}^+, \mathcal{L}^*)$
\end{algorithm}

\newpage

Algorithmus 38 zum Generalisieren eines L-Systems:\\~\\
Initialisierung (Zeile 2-4):
\begin{itemize}
    \item Das zu untersuchendes Tuple bestehend aus zwei Produktionsregeln (Regelpaar) und wird in der Variable $p^*$ gehalten.
    \item L-Systeme, die sich infolge eines Merges geändert haben, werden in $L^+$ und Eingabe-L-systeme in $L^*$ gespeichert.
\end{itemize}
Generalisierung (Zeile 6-13):
\begin{itemize}
    \item $\mathcal{P}$ ist die Menge aller möglichen Regelpaare aus $L^*$.
    \item Das Regelpaar, das beim Merge die geringsten Kosten für die Überführung in die neue Grammatik aufweist, wird in
    $p^*$ gehalten.
    \item Sind diese Kosten positiv, wird die Generalisierung abgebrochen.
    \item Andernfalls werden die Variablen $c^*$ als Delta-Kosten, $C^{old}_g$ und $L^*$ entsprechend gesetzt.
    \item Sollte die Differenz der Überführungskosten positiv sein, wird die Generalisierung abgebrochen.
\end{itemize}

\begin{algorithm}[caption={Generalisieren eines L-Systems mit Gewichtung $w_0$}]
Initialisierung:
    Regelpaar $p^* = \emptyset$
    $\mathcal{L}^* = \mathcal{L}^+$
    $C_g^{old} = C_g(\mathcal{L}^* + \{p^*\}, \mathcal{L}^*)$
Schleife:
    Finde Regelpaar $p^*$ mit minimalen Kosten $C_g(\mathcal{L}^* + \{p_i\}, \mathcal{L}^*), \forall p_i \in \mathcal{P}$
    Wenn $C_g(\mathcal{L}^* + \{p^*\}, \mathcal{L}^*) \geq 0$:
        Breche Schleife ab
    $c^* = C_g(\mathcal{L}^* + \{p^*\}, \mathcal{L}^*) - C_g^{old}$
    $C_g^{old} = C_g(\mathcal{L}^* + \{p^*\}, \mathcal{L}^*)$
    $\mathcal{L}^* = \mathcal{L}^* + \{p^*\}$
    Wenn $c^* > 0$:
        Breche Schleife ab
\end{algorithm}

\newpage

\section{Softwaretechnik}

\subsection*{Extreme Programming}
Eine Fallstudie der Universität Karlsruhe~\cite{muller_2001} untersucht den Einsatz der Softwaretechnik Extreme
Programming (XP) im Kontext der Erstellung von Abschlussarbeiten im Universitätsumfeld.
Hierzu werden folgende Schlüsselpraktiken untersucht:
\begin{itemize}
    \item XP als Softwaretechnik zur schrittweisen Annäherung an die Anforderungen eines Systems
    \item Änderung der Anforderungen an das Systems
    \item Funktionalitäten (Features) werden als Tätigkeiten des Benutzers (User Stories) definiert
    \item Zuerst werden Komponententests (Modultests) geschrieben und anschließend die Features (Test-driven Design)
    \item Keine seperaten Testing-Phasen
    \item Keine formalen Reviews oder Inspektionen
    \item Regelmäßige Integration von Änderungen
    \item Gemeinsame Implementierung (Pair Programming) in Zweiergruppen
\end{itemize}
Aus der Fallstudie geht hervor, dass Extreme Programming einige Vorteile bei der Bearbeitung eines Softwareprojektes
einer Bachelorarbeit bietet.
Zum Einen können sich Anforderungen an das zu erstellende System durch parallele Literaturrecherche ändern, zum
Anderen können Arbeitspakete durch Releases abgebildet werden.
Diese Softwaretechnik wird in der Umsetzung des Softwareprojekts zu dieser Arbeit angewendet.

\begin{figure}[H]
    \centering
    \includegraphics[width=12.8cm]{../images/extreme_programming.png}
    \caption{Zyklische Umsetzung des Softwareprojektes}
\end{figure}

\section{Softwarearchitektur}
Die Gliederung der Inhalte für die Softwarearchitektur erfolgt nach der arc42-Vorlage~\cite{arc42}

\subsection*{Qualitätsziele}
Um die wesentlichen Features des Systems in einem Programm umzusetzen, werden Qualitätsziele in einem Qualitätsbaum
(siehe~\ref{baum}) definiert und Szenarien zugeordnet (Nummerierung).
Die nichtfunktionalen Anforderungen sind nach der DIN-Norm \texttt{DIN 66272} strukturiert.\\~\\
Der Benutzer ist in der Lage eine Verzweigungsstruktur zu erstellen, aus der dann eine oder mehrere ähnliche Strukturen
erzeugt werden können \textbf{(1)}.
Vordefinierte Templates sollen verwendet werden, um die Basisstruktur zu erzeugen \textbf{(2)}.
Dabei soll der Benutzer visuelles Feedback während des Erstellungsprozesses bekommen \textbf{(3)}.
Für den Benutzer besteht die Möglichkeit Parameter zu setzen, die die resultierenden Ähnlichkeitsstrukturen beeinflussen,
um so zu einem akzeptablen Ergebnis zu gelangen \textbf{(4)}.
Es soll nicht möglich sein eine ungültige Verzweigungsstruktur zu erstellen \textbf{(5)}.
Der Erstellungsvorgang soll jeder Zeit neu begonnen werden können, ohne das Programm neu starten zu müssen \textbf{(6)}.
Weiter sollte das Programm robust gegenüber Benutzereingaben sein \textbf{(7)}.
Eine intuitive Nutzung des Programms soll gegeben sein, sodass der Benutzer wenig Zeit für die Nutzung der Bedienelemente
aufbringen muss \textbf{(8)}.
Das Erzeugen der Struktur aus Kapitel~\ref{eval} (Evaluierung) soll ohne händische Nachbildung möglich sein \textbf{(9)}.\\
Zur Nachvollziehbarkeit sollten die verschiedenen L-System-Varianten in der Systemkonsole ausgegeben werden \textbf{(10)}.
Der Generierungsprozess nach dem Erstellen der Basisstruktur sollte nicht länger als vier Sekunden in Anspruch nehmen \textbf{(11)}.
Über ein Textfeld sollte der Benutzer auch Fließkommazahlen als parameter festlegen können \textbf{(12)}.
Ein Entwickler soll zur Manipulation von L-Systemen weitere Algorithmen hinzufügen können \textbf{(13)}.
Das Programm sollte unter eine Java-Laufzeitumgebung auf mehreren Systemen nutzbar sein \textbf{(14)}.

\subsection*{Lösungsstrategie}
Zur Erreichung der nichtfunktionalen Qualitätsziele werden folgende Architekuransätze umgesetzt.
Die funktionalen Ziele werden durch die in Kapitel~\ref{probleme} vorgestellten Lösungsansätze abgedeckt.
\begin{center}
    \begin{tabular}{l|l}
        \textbf{Qualitätsziel} & \textbf{Architekturansatz} \\
        \hline \\
        Funktionalität &
        \begin{minipage}[t]{0.8\textwidth}
            Einlesen von Templates\\
            Grafische Benutzerschnittstelle zur Anordnungen von Templates zu einer Verzweigungsstruktur\\
            Generieren der Baumstruktur\\
            Verarbeiten von L-Systemen
        \end{minipage} \\
        \\ \hline \\
        Zuverlässigkeit &
        \begin{minipage}[t]{0.8\textwidth}
            Durch den User vorgegebene Parameter, wie Anzahl erzeugter Ähnlichkeitsstrukturen, maximale Ausführungstiefe
            des resultieren L-Systems und Gewichtungsparameter der Algorithmen\\
            Definieren der grafischen Bedienelemente, sodass kein unerwartetes Verhalten auftritt
        \end{minipage} \\
        \\ \hline \\
    \end{tabular}
\end{center}
\begin{center}
    \begin{tabular}{l|l}
        \textbf{Qualitätsziel} & \textbf{Architekturansatz} \\
        \hline \\
        Benutzbarkeit &
        \begin{minipage}[t]{0.8\textwidth}
            Zeichnen der Verzweigungssstruktur und voläufige Änderungen von Parametern während der Erstellung\\
            Zurücksetzen des Erstellungsprozesses durch den Benutzer\\
            Intuitive Benennung von Schaltflächen\\
            Schaltfläche zur Erzeugung des Beispiels aus Kapitel~\ref{eval}\\
            Steuern des Generierungsprozesses durch Setzen von Parametern durch den Benutzer
        \end{minipage} \\
        \\ \hline \\
        Effizienz &
        \begin{minipage}[t]{0.8\textwidth}
            Ausgabe der Teilschritte in der Systemkonsole
        \end{minipage} \\
        \\ \hline \\
        Änderbarkeit &
        \begin{minipage}[t]{0.8\textwidth}
            Das Nutzen des Pipeline Design Patterns soll das Erweitern des Systems durch
            Hinzufügen weiterer Teilschritte (Pipes) erleichtern.
            Trennung der grafischen Oberfläche und der Logik durch Aufbauen des Szenengraphen über ein
            XML-Dateiformat
        \end{minipage} \\
        \\ \hline \\
        Übertragbarkeit &
        \begin{minipage}[t]{0.8\textwidth}
            Erstellung einer ausführbaren Java-Archiv-Datei\\
            Sowohl eine sinnvolle Aufteilung von Funktionalitäten auf Dateien und Software-Pakete, als
            auch effiziente Datenkapselung und geschlossene Informationskontexte sorgen für Modularität des
            Programms
        \end{minipage} \\
        \\ \hline
    \end{tabular}
\end{center}

\newpage

\subsection*{Kontextabgrenzung}
Die Systemgrenzen werden zum Einen durch die Interaktion mit dem Benutzer, zum Anderen durch die Interaktion mit
dem Dateisystem des Host-Systems und dem Zugreifen und Lesen der Template-Dateien definiert.
Hierbei wird die Erstellung der Basisstruktur als nicht-technische Interaktion und das Einlesen der Dateien als
technische Interaktion gesehen.
\begin{figure}[H]
    \centering
    \includegraphics[width=6.2cm]{../images/Fachlicher_Kontext.PNG}
    \caption{System und Systemumgebung}
\end{figure}
\begin{figure}[H]
    \centering
    \includegraphics[width=10cm]{../images/Technischer_Kontext.PNG}
    \caption{Interaktion zwischen System und Systemumgebung}
\end{figure}

\subsection*{Bausteinsicht}
Der Benutzer interagiert über das User Interface mit dem Subsystem GUI, das sich mit der Visualisierung, dem Aufbau der
Eingabestruktur und dem Anlegen einer internen Baumtopologie beschäftigt.
Die Pipeline zum Erzeugen der Ausgabestrukturen beginnt mit dem Inferieren eines L-Systems aus der benutzerdefinierten
Struktur und gibt das erzeuge Ersetzungssystem an die folgenden Komponenten weiter.
Hierbei gilt der Pipeline-Kontext, welcher innerhalb der Pipeline an den jeweils nächsten Schritt weitergegeben und
dort aktualisiert wird, als Eingabe der Pipeline.
Hat die Estimator-Komponente eine Verteilung über vom Benutzer angelegte Transformationsparameter angelegt, gilt
der Kontext als Ausgabe der Pipeline.

\underline{Ebene 1}
\begin{figure}[H]
    \centering
    \includegraphics[width=12cm]{../images/Bausteinsicht_Ebene_1.PNG}
    \caption{Subsysteme mit fachlichen Abhängigkeiten}
\end{figure}

Betrachtet man die GUI-Komponente genauer, setzt sich diese aus vier Subsystemen zusammen.
Das Application-Modul beschreibt den Einstigspunkt des Programms.
Die grafische Benutzerschnittstelle wird aus einem View (Scene graph) mit zugeförigem Kontroller (Model)
aufgebaut.
Der Kontroller stellt die Logik für das View zur Verfügung.
Neben der Erstellung der Basisstruktur, wird die Baumstruktur in einer seperaten Komponente umgesetzt.

\underline{Ebene 2}
\begin{figure}[H]
    \centering
    \includegraphics[width=9cm]{../images/Bausteinsicht_Ebene_2.PNG}
    \caption{Subsystem GUI}
\end{figure}

\subsection*{Laufzeitsicht}
Aus der nachfolgenden Abbildung geht hervor, wie sich der Informationsfluss zwischen den einzelnen Subsystemen verhält.
Wird der Systemprozess einmal durchlaufen, besteht die Möglichkeit die Ausführung des generalisierten L-Systems mit
erneuter Vergabe der Transformationsparameter aus der Häufigkeitsverteilung der Estimator-Komponente zu starten.
Der Ablauf des Systems setzt sich wie folgt zusammen:

\begin{figure}[H]
    \centering
    \includegraphics[width=14cm]{../images/Laufzeitsicht.PNG}
    \caption{Laufzeitsicht}
\end{figure}

\subsection*{Verteilungsicht}
Die Ausführung des Programms wird durch ein einfaches Startskript zur Verfügung gestellt.
Die folgende Abbildung der Verteilungssicht stellt lediglich die Ausführung auf einer Windows-Maschine dar.

\begin{figure}[H]
    \centering
    \includegraphics[width=10cm]{../images/Verteilungssicht.PNG}
    \caption{Infrastruktur Windows-PC}
\end{figure}

\subsection*{Datenstrukturen}
Das wesentliche Datenmodell ist als UML-Diagramm modelliert (siehe Seite~\pageref{anhang} Anhang).\\~\\
Die während der Strukturierung der Templates zu einer Verzweigungsstruktur entstehenden Baumstuktur wird in~\ref{baum} gezeigt.
\texttt{TreeNode} implementiert das \texttt{Iterable}-Interface, sodass durch die Überladung der \texttt{iterate}-Funktion
ein Iterator zurückgegeben werden kann, der eine Iterierung über die Baumtopologie möglich macht.
Dieser Iterator ist durch die Klasse \texttt{TreeNodeIterator}, die das \texttt{Iterator}-Interface implementiert, definiert.
Die überladene \texttt{next}-Funktion beschreibt eine Iterierung des Baums durch eine Breitensuche.\\~\\
~\ref{pipeline} beschreibt den Generierungsprozess im Pipeline Design Pattern.
Jeder Teilschritt des Prozesses ist durch eine Klasse, die das \texttt{Pipe}-Interface nutzt, definiert.
Die überladene \texttt{process}-Funktion nimmt den \texttt{PipelineContext} entgegen, verändert diesen und gibt ihn
anschließend zurück.
Das \texttt{PipelineContext}-Objekt hält die Daten, die die eizelnen Pipes benötigen und zurückgeben (Pipeline-Kontext).
Die Pipes lassen sich in einer festen Reihenfolge durch die überladene \texttt{pipe}-Funktion eines
\texttt{Pipeline}-Objekts in die Pipeline einreihen.
Diese wird durch \texttt{execute} ausgeführt.\\~\\
Die verschiedenen Algorithmen zur Manipulation von L-Systemen werden in~\ref{tool} durch verschiedene Objekte definiert.
Sie werden in der entsprechenden Pipe erstellt und durch eine Funktion ausgeführt, die das veränderte L-System
zurückgibt.
Im Konstruktor werden benötigen Daten aus dem Pipeline-Kontext an den Algorithmus übergeben.

% @author Arian Helberg

\chapter{Implementierung}

\section{Pakete?}

Die Implementierung des Programms setzt sich aus folgenden Teilschritten zusammen:
\begin{itemize}
    \item Erstellung der \textit{GUI} (Paket gui, tree) mit
    \begin{itemize}
        \item UI-Elementen
        \item Render-Canvas
        \item Dateianbindung der Templates
        \item Erstellung der repräsentativen Baumstruktur\\ (\textit{treeGenerator}, Paket tree)
    \end{itemize}
    \item Implementierung der Subsysteme als Pipes
    \begin{itemize}
        \item \textit{Inferer} (Paket grammar): Ableiten eines kompakten L-Systems aus einer Baumstruktur
        \item \textit{Generalizer} (Paket grammar): Generieren eines generalisierten L-Systems anhand eines
        "`kleinen"' L-Systems
        \item \textit{Randomizer} (Paket grammar): Erzeugung von L-Systemen, die der erstellen
        Verzweigungsstruktur "`ähnlich"' sind
    \end{itemize}
    \item Komponenten- und Systemtests
\end{itemize}
\begin{center}
    \begin{tabular}{l|l}
        \textbf{Subsystem} & \textbf{Umsetzung} \\
        \hline \\
        GUI &
        \begin{minipage}[t]{0.8\textwidth}
            JavaFX als Framework zur Erstellung von grafischen und interaktiven Inhalten.
            Erstellung der Baumstruktur über dynamisches Erzeugen von Konten während der Strukturierung der
            Verzweigungsstruktur
        \end{minipage} \\
        \\ \hline \\
        Inferer &
        \begin{minipage}[t]{0.8\textwidth}
            Algorithmus zum Iterieren maximaler Sub-Bäume und deren Reduzierung mittels Ersetzung durch Symbole
            und der zugehörigen Produktionsregel, bis eine Kostengrenze, die durch eine Kostenfunktion abgebildet
            werden kann, erreicht ist
        \end{minipage} \\
        \\ \hline \\
        Generalizer &
        \begin{minipage}[t]{0.8\textwidth}
            Algorithmus zum Erweitern eines L-Systems um nicht-deterministischer Regeln und Erkennen rekursiver
            Strukturen
        \end{minipage} \\
        \\ \hline \\
        Randomizer &
        \begin{minipage}[t]{0.8\textwidth}
            ddd sfevdhbnreaydtydbfsdxc
        \end{minipage} \\
        \\ \hline
    \end{tabular}
\end{center}

\section{Entscheidungen}
\underline{Mutable or Immutable Objects?}\\~\\
\underline{Risiken}\\~\\
\underline{Qualitätsmerkmale}\\~\\
\underline{Alternativen}\\~\\
\underline{Aufwand der Implementierung}

\section{Technologien}

\begin{itemize}
    \item Programmiersprache: Java Version 11 mit
    \begin{itemize}
        \item JavaFX Version 15 (openjfx)
    \end{itemize}
    \item Build-Management-Tool: Gradle\cite{gradle} Version 6.7
    \item Versionskontrolle: Github Repository\cite{github} via Git\cite{git}
    \item IDE: JetBrains IntelliJ IDEA\cite{idea} 2020.2.2 (Ultimate Edition)
    \item Betriebssystem: Microsoft Windows 10 Pro 64 Bit
    \item User Story Map: Trello Board\cite{trello}
\end{itemize}

\section{Hardware}

\begin{itemize}
    \item Prozessor: Intel Core i5-3570K CPU @ 3.40GHz
\end{itemize}
% @author Arian Helberg

\chapter{Evaluierung}
\label{eval}
Die Synthese von Verzweigungstrukturen wird in den vorangegangenen Kapiteln konzeptionell betrachtet
und in einem Softwareprojekt umgesetzt.
Im Folgendem werden Teilaspekte an einem fortlaufenden Beispiel evaluiert und bewertet.
Diese Aspekte umfassen:
\begin{itemize}
    \item das Nutzen von Templates
    \item die Erstellung und Visualisierung von Verzweigungsstrukturen
    \item Aufbau einer Baumstruktur
    \item Extrahierung von Regeln und Mustern
    \item Komprimierung von L-Systemen
    \item Erweiterung von L-Systemen
    \item Synthese von Ähnlichkeitsstrukturen
    \item Auswirkung der Gewichtungsparameter
\end{itemize}

\subsection*{Templates}
Die Nutzung von Templates als Terminale einer Grammatik findet Anwendung in vielen wissenschaftlichen Arbeiten.
\citeauthor{aliaga_2016} liefert hierzu eine umfassende Übersicht~\cite{aliaga_2016}.
Mit der Verwendung einer allgemeinen Repräsentation mithilfe von \textit{turtle}-Befehlen, wird eine
Wiederverwendbarkeit der genutzten Templates sichergestellt.
Die in dieser Arbeit erstellte Software nutzt ein minimalistisches System zum Einlesen der Templates aus
einer textbasierten Datei.
Die Forschung zur inversen prozeduralen Modellierung zeigt wiederum den Einsatz von
neuronalen Netzen als vielversprechende Methode Strukturen zu erkennen und Regeln abzuleiten.\\
Während neuronale Netze eine Eingabestruktur analysieren und in einer baumähnlichen Struktur organisieren,
knüpft diese Arbeit hier an und nutzt aus Praktikabilitätsgründen stattdessen eine benutzergeführte Anordnung
von Templates zu einer Verzweigungsstruktur.\\~\\
\underline{Beispiel}: Die Template-Zeichenketten haben folgende Form:
\begin{figure}[H]
    \centering
    \begin{csource}
    FX
    F-FX
    F+FX
    F[-FX]+FY
    F[-FX]FY
    F[FX]+FY
    F[-FX][FY]+FZ
    F[+FX]F[-FY]FZ
    \end{csource}
    \caption{templates.txt}
\end{figure}
\textit{F}, \textit{-}, \textit{+}, \textit{[} und \textit{]} sind aus dem Turtle-Algorithmus bekannte
Befehlssymbole.
Alle anderen Symbole (z.B. \textit{X, Y, Z}) stellen Verzweigungsvariablen dar, um anzuzeigen, an welcher Stelle
der Templatestruktur eine neue Verzweigung abgehen kann.

\subsection*{Erstellung und Visualisierung}
Durch die Ausführung bestimmter \textit{turtle}-Befehle der template-basierten Struktur, wird diese visualisiert.
Deshalb bietet es sich an, einfache grafische Elemente zu nutzen, um Verzweigungsstrukturen sichtbar zu machen
(z.B. Canvas).
Das Programm nutzt stattdessen interaktive Elemente innerhalb eines JavaFX Pane gegenüber statischen Elementen, um
die Strukturen zu zeichnen.
Der Prozess der Erstellung ist benutzergeführt und muss somit interagierbar sein.
So können Elemente genutzt werden, mit denen der Benutzer kommunizieren kann (z.B. klickbare Kreise).
Es ist nun möglich Verzweigungsstrukturen in einem simplen Arbeitsablauf zu erstellen.

\newpage

\underline{Beispiel}: Aus der Erstellung der Verzweigungsstruktur ergibt sich folgende Abbildung:
\begin{figure}[H]
    \centering
    \includegraphics[width=12cm]{../images/evaluierung_inferrieren.png}
    \caption{Grafische Benutzeroberfläche nach der Erstellung einer Basisstruktur}
\end{figure}

\subsection*{Aufbau einer Baumstruktur}
Baumähnliche Strukturen eignen sich gut, um Tansformationsparameter und topolo-gische Anordnung von Templates
datenstrukturell zu trennen.\\
Ein ganzheitlicher Ansatz zur Analyse von Verzweigungsstrukturen ist die Kombina-tion aus Strukturtopologie und
räumlichen Transformationen.
Ein weiterer Ansatz ist die separate Betrachtung von Topologie und Transformation.
Beide Ansätze sind derzeit Gegenstand der Forschung.\\
Das von \citeauthor{benes_2011} vorgestellte System nutzt beispielsweise einen ganzheitlichen Ansatz, um sog.
\textit{Guides} zu erstellen, welche Teilsysteme der Eingabestruktur beschreiben.
Hier unterscheiden sich Strukturen, die zwar identische Verzweigungen aufweisen, jedoch deren
Transformationsparameter (z.B. der Winkel von Verzweigungen) voneinander abweichen.
Auch in der Arbeit von \citeauthor{stava_2010} zeigt sich eine Organisation von
"`Clustern"' mit Transformationen, die in eine Signifikanzbewertung einfließen~\cite{stava_2010}.

\newpage

Bei einer separaten Betrachtung von Topologie und Transformation zeigen \citeauthor{nishida_2016} und \citeauthor{guo_2020},
dass es sinnvoll ist zwei spezialisierte, neuronale Netze zu verwenden, da die räumlichen Transformationen das Erkennen
der Topologie nicht signifikant beeinflussen~\cite{nishida_2016, guo_2020}.\\
Diese Arbeit legt den Fokus auf die datenstrukturelle Trennung von \mbox{Transformationen} und topologischer
Anordnung und erstellt somit eine Baumstruktur, die Template-Instanzen als Knoten und
räumliche Transformationen als eingehende Kanten darstellt.\\~\\
\underline{Beispiel}: Aus der Eingabestruktur ergibt sich folgender Baum (die räumlichen Transformationen sind hier nicht visualisiert):
\begin{figure}[H]
    \centering
    \includegraphics[width=12cm]{../images/evaluierung_inferrieren_baum.png}
    \caption{Baumstruktur der erstellten Verzweigungsstruktur}
\end{figure}

\subsection*{Extrahieren von Regeln und Mustern}
Die in~\ref{alg2} vorgestellte Methodik zum Inferieren eines L-Systems aus einer Baumstruktur
stellt einen Algorithmus vor, der auf die spezielle Baumstruktur zugeschnitten ist.
Die L-Systeme entsprechen lediglich der Eingabestruktur und beinhalten keine Transformationen.\\~\\
\underline{Beispiel}: Ausgeführte Ersetzungssysteme werden über einen JavaFX Dialog visualisiert:
\begin{figure}[H]
    \centering
    \includegraphics[width=8cm]{../images/evaluierung_inferrieren_lsystem.png}
    \caption{Inferiertes L-System}
\end{figure}
Die Zeichenkettenrepräsentation des L-Systems ($\mathcal{L}=\{M,\omega,R\}$) lautet:
\begin{csource}
LSystem{
    [F, S, A, B, C, D, E, G, H, I, J, K, L, M, N, O, P, Q, R, T, U, V, W, X, Y, Z, a, b, c, d, e, f, g, h, i, j, k],
    S,
    [S -> A, A -> F[-FB][FC]+FD, B -> F+FE, C -> F[FG]+FH, D -> F[FI]+FJ, E -> F[-FK]+FL, G -> _, H -> F-FM, I -> F[+FN]F[-FO]FP, J -> F[FQ]+FR, K -> _, L -> F[-FT]+FU, M -> _, N -> _, O -> _, P -> F[-FV]+FW, Q -> F[-FX]+FY, R -> _, T -> F+FZ, U -> _, V -> F+Fa, W -> _, X -> F+Fb, Y -> _, Z -> F[+Fc]F[-Fd]Fe, a -> F[+Ff]F[-Fg]Fh, b -> F[+Fi]F[-Fj]Fk, c -> _, d -> _, e -> _, f -> _, g -> _, h -> _, i -> _, j -> _, k -> _]
}
\end{csource}

Das inferierte L-System zeigt, dass die Eingabestruktur richtig in eine Grammatik überführt wurde.
Dabei steht der Unterstrich (\_) für das leere Wort $\varepsilon$ und damit für die leeren Knoten
der aufgebauten Baumstruktur.

\subsection*{Komprimieren des L-Systems}
Die Zeichenkettenrepräsentation des inferierten L-Systems zeigt einige Redundanzen auf.
Zum Beispiel bilden
\begin{csource}
    P -> F[-FF+FF[+F]F[-F]F]+F
\end{csource}
und
\begin{csource}
    Q -> F[-FF+FF[+F]F[-F]F]+F
\end{csource}
das gleiche Muster.
Um solche Redundanzen zu entfernen, wird das L-System in der vorgestellten Komprimierungs-Pipe reduziert.
Hierbei werden identische, maximale\\Unterbäume und deren leere Kindknoten zusammengefasst.
Der Gewichtungsparameter für das Beispiel wird auf 0.5 gesetzt.
Es ergibt sich folgender Baum:
\begin{figure}[H]
    \centering
    \includegraphics[width=14cm]{../images/compressed_tree.png}
    \caption{Komprimierter Baum}
\end{figure}

Die Baumstruktur zeigt, dass der Komprimierungsalgorithmus die maximalen Subbäume erfolgreich reduziert.
Das komprimierte L-System setzt sich wie folgt zusammen:
\begin{csource}
LSystem{
    [F, S, A, B, C, D, E, G, H, I, J, K, L, M, N, O, P],
    S,
    [S -> A, A -> F[-FB][FC]+FD, B -> F+FE, C -> F[FG]+FH, D -> F[FI]+FJ, E -> F[-FK]+FL, G -> _, H -> F-FM, I -> F[+FN]F[-FO]FP, J -> F[FL]+FG, K -> _, L -> F[-FF+FF[+F]F[-F]F]+F, M -> _, N -> _, O -> _, P -> F[-FF+FF[+F]F[-F]F]+F]
}
\end{csource}

\subsection*{Generalisieren des L-Systems}
Der vorgestellte Generalisierungsalgorithmus verändert das L-System der Eingabestruktur, indem
Produktionsregeln miteinander verbunden und mit einer Wahrscheinlichkeit versehen werden.
Der Gewichtungsparameter wird für das Beispiel auf 0.5 gesetzt.
Es ergibt sich das finale L-System:
\begin{csource}
LSystem{
    [F, S, E, G, K, M, N, A],
    S,
    [S -> A, E -> F[-FK]+FA, G -> _, K -> _, M -> _, N -> _, A -> F-FM, A -> F[-FA][FA]+FA, A -> F[FG]+FA, A -> F[FA]+FG, A -> F+FE, A -> F[+FN]F[-FA]FA, A -> F[-FF+FF[+F]F[-F]F]+F, A -> F[FA]+FA, A -> _]
}
\end{csource}

Die Produktionsregelmenge zeigt einige Regeln auf, die dasselbe Ziel haben.
Sie werden bei ihrer Anwendung zufällig ausgewählt.


\subsection*{Synthese}
Führt man das generalisierte L-System aus, ergeben sich die Ähnlichkeitsstrukturen.
Eine Evaluierung wird stichprobenartig durchgeführt, erweist sich jedoch als schwierig, da
es bei der Erstellung der Ausgabestrukturen um Wahrscheinlichkeiten beim Auftreten gewisser
Muster handelt.
Da die Algorithmen eine akzeptable Lösung eines schwierigen Problems liefern sollen,
ist die Bewertung der Ergebnisse subjektiv.
\begin{figure}[H]
    \centering
    \includegraphics[width=14.5cm]{../images/synthesis.png}
    \caption{Synthetisierte Verzweigungsstrukturen}
\end{figure}

Vergleicht man die Eingabestruktur mit den synthetisierten Strukturen, weisen diese
gewisse Eigenschaften auf (siehe folgende Abbildung):
\begin{itemize}
    \item Wiederholtes Auftreten einer Struktur, die beim Synthetisieren als maximaler\\Unterbaum
    erkannt wurde (rot)
    \item Anwendung von benutzerdefinierten Transformationsparametern (blau)
    \item Ähnliche Häufigkeiten (grün)
    \item Isoliertes Auftreten von Templates des maximalen Subbaums, die über den Unterbaum
    hinaus einzeln vorkommen (gelb)
\end{itemize}

\begin{figure}[H]
    \centering
    \includegraphics[width=14.5cm]{../images/synthesis_marks.png}
    \caption{Untersuchung der synthetisierten Verzweigungsstrukturen}
\end{figure}

\subsection*{Auswirkung der Gewichtungsparameter}
Der Parameter $w_l$ des vorgestellten Komprimierungsalgorithmus gewichtet eine Funktion,
welche die Kosten eines L-Systems ermittelt, sodass der Algorithmus L-Systeme mit unterschiedlich
großer Produktionsregelmenge anders behandelt.
Hierbei bewertet ein Wert von $0.0$ eine große Produktionsregelmenge höher.
Entspricht $w_l$ einem Wert von $1.0$, wird eine kleinere Produktionsregelmenge überbewertet.
Der Durchschnittswert $0.5$, der im Beispiel verwendet wird, sorgt für eine moderate Menge an
Produktionsregeln bei moderater Länge der einzelnen Regel.
Der im Generalisierungsalgorithmus vorgestellte Gewichtungsparameter $w_0$ legt fest, inwiefern
die Länge oder \textit{String Edit Distance} zweier Grammatiken von Bedeutung sind.
Bei einem Wert von $0.0$ überbewertet der Algorithmus die Metrik zur Berechnung der Anzahl Operationen,
um eine Grammatik in eine zweite zu überführen.
Der Wert $1.0$ entspricht einer Überbewertung der Längenmetrik.
Im Beispiel wird der Durchschnittswert $0.5$ verwendet.

$w_l$ und $w_0$ sind im Programm als \textit{Rula application ratio} und \textit{Merge application ratio}
bezeichnet und können vom Benutzer vor der Generierung festgelegt werden.
% @author Arian Helberg

\chapter{Ausblick}
\ldots
% Add additional chapters here

%\bibliographystyle{plain}
\bibliographystyle{dinat}
\bibliography{literature}

% Appendix
\appendix
\input{appendix/example_appendix}

\IGlossary

\Istatement

\end{document}
