% !TEX root = ../thesis.tex
%
% configurations
%

% text field
%-> replace supervisor names with correct ones
\firstSupervisor{Prof. Dr. Philipp Jenke}
\secondSupervisor{Prof. Dr. Michael Neitzke}

% text field
%-> replace title with your thesis title
\thesisTitle{Template-basierte Synthese von Verzweigungsstrukturen mittels L-Systemen}
\thesisTitleEN{Template-based synthesis of branching structures using L-Systems}

% text field
%-> replace the key words with your own key words
\keywordsDE{Verzweigungsstruktur, 2D Generierung, L-System, Template, Prozedurale Modellierung, Inverse Prozedurale Modellierung, Baumstruktur, Formale Grammatik, Ähnlichkeit}
\keywordsEN{Branching Structure, 2D Generation, L-System, Template, Procedural Modeling, Inverse Procedural Modeling, Tree Structure, Formal Grammar, Similarity}

% text field
%-> replace the text with a description of the thesis
\abstractDE{
Prozedurale Modellierung beschreibt effiziente Methoden zur Erzeugung einer Vielzahl
an Modellen nach bestimmten Regeln. Die Erstellung eines Systems zur Umsetzung solcher Methoden ist durch
einen Mangel an Kontrolle und einer geringen Vorhersagbarkeit der Ergebnisse erschwert.
Diese Bachelorarbeit präsentiert ein System zur Synthese von Verzweigungsstrukturen, die einer benutzerdefinierten
Strukur ähnlich sind. Dabei wird gezeigt, wie sich aktuelle Ansätze der Forschung in einem Programm umsetzen lassen.
Templates werden eingelesen und vom Benutzer zu einer Basisstruktur organisiert.
Anschließend wird ein L-System über die Topologie dieser Struktur inferiert, komprimiert und dann generalisiert.
Nach der Interpretation des L-Systems können vom Benutzer gesetzte Transformationsparameter aus einer Häufigkeitsverteilung
angewendet werden. Zum Schluss wird die resultierende Verzweigungsstruktur visualisiert.
}
\abstractEN{
Procedural modeling describes efficient methods for creating various models according to certain rules.
The creation of a system to implement such methods is complicated due to a lack of control and a low predictability of the results.
This bachelor thesis presents a system for synthesizing branching structures that are similar to custom structures created by a user.
It is shown how current research approaches can be implemented.
A basic structure is build from templates by the user.
Subsequently an L-System is inferred from the topology of this structure, compressed and then generalized.
User-defined transformation parameters from a frequency distribution can be applied to the interpretation of this L-System.
Finally, the resulting branching structure is visualized.
}

% text field
%-> replace jon with your name
\thesisAuthor{Adrian Helberg}

% text field
%-> enter the submission date
\submissionDate{\today}

% switch - uncomment only one
%-> uncomment NDA or public
%\NDA{yes}
\NDA{no}

%-> uncomment cover or cover Corporate Design 2017
\Cover{CD2017}
%\Cover{CD2017NoLogo}
%\Cover{Std2018}

% switch - uncomment only one
%-> uncomment to show list of figures or not
\ListOfFigures{yes}
%\ListOfFigures{no}

% switch - uncomment only one
%-> uncomment to show list of tables or not
\ListOfTables{yes}
%\ListOfTables{no}

% switch - uncomment only one
%-> uncomment to show list of accronyms or not
\ListOfAccronyms{yes}
%\ListOfAccronyms{no}

% switch - uncomment only one
%-> uncomment to show list of symbols or not
\ListOfSymbols{yes}
%\ListOfSymbols{no}

% switch - uncomment only one
%-> uncomment to show list of glossary entries or not
\Glossary{yes}
%\Glossary{no}

% switch - uncomment only one
%-> uncomment the study course you are in
%\studycourse{ITS}
%\studycourse{TI}
\studycourse{AI}
%\studycourse{WI}
%\studycourse{EI}
%\studycourse{REE}
%\studycourse{BMT}
%\studycourse{MAI}
%\studycourse{MIK}
%\studycourse{MA}
