%! Author = Adrian Helberg
%! Date = 21.09.2020

% Preamble
\documentclass[11pt]{article}
% Packages
\usepackage{ngerman}
\usepackage{amsmath}
\usepackage{url}
\usepackage{enumitem,amssymb}
\newlist{todolist}{itemize}{2}
\setlist[todolist]{label=$\square$}
\usepackage{pifont}
\newcommand{\cmark}{\ding{51}}%
\newcommand{\xmark}{\ding{55}}%
\newcommand{\done}{\rlap{$\square$}{\raisebox{2pt}{\large\hspace{1pt}\cmark}}%
\hspace{-2.5pt}}
\newcommand{\wontfix}{\rlap{$\square$}{\large\hspace{1pt}\xmark}}

\title{\textbf{Exposé} zur Bachelorarbeit von Adrian Helberg bei Prof. Dr. Jenke}

% Document
\begin{document}

    \maketitle

    \section{Problemstellung}

    Effizientes Objektdesign und -modellierung sind entscheidende Kernkompetenzen in verschiedenen Bereichen der
    digitalen Welt.
    Da die Erstellung geometrischer Objekte unintuitiv ist und ein großes Maß an Erfahrung und Expertise
    erfordert, ist dieses stetig wachsende Feld für Neueinsteiger nur sehr schwer zu erschließen.
    Die Forschung liefert hierzu einige Arbeiten zu prozeduraler Modellierung, um digitale Inhalte schneller und
    automatisiert zu erstellen.
    Die Bachelorarbeit soll sich auf die Erkenntnisse einer Basisquelle~\cite{basisquelle} stützen und im
    dreidimensionalen Kontext die inverse prozedurale Modellierung von Verzweigungsstrukturen beschreiben.
    Das in der Basisquelle beschriebene System generiert zweidimensionale Objekte mit Verzweigungsstrukturen.
    Hierzu soll ein prototypische Implementierung zur Überführung des Algorithmus vom zweidimensionalen in den
    dreisimensionalen Raum implementiert werden.\\~\\
    \textit{Die Einführung des Themas könnte wie folgt aussehen}:

    \subsection{Einführung}
    Mit der Digitalisierung der Welt erlangt die Erstellung von digitalen Inhalten, wie 3D Modelle für Computerspiele,
    Webdesigns oder Visualisierung von Architektur, zunehmend an Bedeutung.
    Darum werden Verfahren gesucht, um Objekte dieser Felder formal zu beschreiben und somit kodifizierbar
    (\textit{engl. codify}) zu machen.
    Hierbei bilden sich folgende Klassen heraus:
    \begin{itemize}
        \item \textbf{Modellierung}: Um einen physikalischen Körper in ein digitales Objekt zu überführen, wird mithilfe
        von Abstraktion (oder Modellierung) ein mathematisches Modell erstellt, dass diesen Körper formal beschreibt.
        3D Grafiksoftware, wie Blender~\cite{blender}, wird genutzt um geometrische Körper zu modellieren, texturieren
        und zu animieren.
        \item \textbf{Prozedurale Modellierung}: \textit{"`It encompasses a wide variety of generative techniques that
        can (semi-−)automatically produce a specific type of content based on a set of input
        parameters."'}~\cite{1} \\
        "`Prozedurale Modellierung beschreibt generative Techniken, die \\(semi-)automatisch spezifische, digitale
        Inhalte anhand von deskriptiven Parametern erzeugen"' (\textit{Übersetzt durch den Autor})
        \item \textbf{Inverse prozedurale Modellierung (IPM)}:\textit{Aliaga et al.}~\cite{2}
        spricht bei der inversen prozeduralen Modellierung von dem Finden einer prozeduralen Repräsentation von
        Strukturen bestehender Modelle.
    \end{itemize}
    Die Modellierung mithilfe von Grafiksoftware ist eine vergleichbar händische, langwierige Erstellung von
    Objekten.
    Hierbei hat der Designer (= Modellierer) die volle Kontrolle über die Strukturen des Objektes.\\
    Bei der prozeduralen Modellierung werden spezifische Strukturen eines zu erstellenden physikalischen Objektes
    generalisiert und meist über eine Grammatik und globale Parameter abgebildet.
    Während bei der klassischen Modellierung die menschliche Intuition und bei der prozeduralen Modellierung eine
    parametrisierte Grammatik vorausgesetzt wird, arbeitet IPM mit bestehenden Modellen und extrahiert ("`lernt"')
    die Strukturen des Objektes, die automatisch in eine formale Grammatik überführt werden können.
    Die Generierung von prozeduralen Modellen ist ein wichtiges, offenes Problem~\cite{2}.\\
    Aktuelle Ansätze sind:
    \begin{itemize}
        \item Segmentierung von geometrischen Objekten in Ähnlichkeitsgruppen, um Muster (\textit{engl. Patterns}) zu
        erkennen und
        \item Kontrollierte Generierung durch Finden optimaler Prameter und Regeln
    \end{itemize}

    \subsection{Vorangehende Arbeiten}
    \textit{Folgende Stichworte werden später ausführlich erklärt und die Quellen markiert}:
    \begin{itemize}
        \item Prozedurale Modellierung
        \item Inverse Prozedurale Modellierung
        \item Formale Grammatik (parametrisiert und probabilistisch)
        \begin{itemize}
            \item L-Systeme (spezielle, formale Grammatik zur beschreibung von natürlicher Vegetation)
        \end{itemize}
        \item Neurale Netzwerke zum Erkennen von Objekten und Strukturen
    \end{itemize}


    \section{Ziele der Arbeit}
    \begin{itemize}
        \item Erarbeitung eines Software-/Harwarestacks
        \item Erstellung eines Projektplans
        \item Herausarbeiten einer Softwarearchitektur
        \item Aufsetzen eines "`Bachelortagebuchs"'
        \item Prototypische Entwicklung eines Systems zur Erstellung "`ähnlicher"' 3D-Objekte anhand eines
        vorgefertigten Input-Modells mit grafischer Oberfläche
        \item Schriftliche Ausarbeitung der Ergebnisse
    \end{itemize}

    \section{Ablauf}
    \begin{itemize}
        \item Zweiwöchige Meilensteine mit Besprechungen (Videokonferenz, Treffen)
        \item Phasen:
        \begin{itemize}
            \item Vorbereitung
            \item Literaturstudium
            \item Problemstudium
            \item Praktische Arbeit
            \item Schriftliche Arbeit
        \end{itemize}
    \end{itemize}

    \section{Organisatorisches}
    \begin{itemize}
        \item 12. Oktober 2020: Anmeldung der Bachelorarbeit (?)
        \item Online Repository~\cite{github}
    \end{itemize}

    \section{Voraussetzungen}
    \ldots

    \newpage

    \section{Zeitplan}
    \begin{todolist}
        \item[\done] bis 29.08.2020: Erarbeitung Basisquelle
        \item[\done] 14.09.2020: Kennenlerngespräch, erstes Themengespräch
        \item bis 28.09.2020: Version 0.1 des Exposés für die Bachelorarbeit
        \item bis 12.10.2020: Literaturrecherche, Mindmap der untersuchten Quellen
        \item bis 16.10.2020: Ausarbeitung des Software-/Hardwarestacks
        \item bis 26.10.2020: Erstellung des Projektplans
        \item ab 26.10.2020: Bearbeitung des praktischen Teils der Bachelorarbeit, paralleler Beginn des
        schriftlichen Teils
    \end{todolist}

    ~\nocite{*}
    \bibliography{literature}
    \bibliographystyle{plain}

\end{document}