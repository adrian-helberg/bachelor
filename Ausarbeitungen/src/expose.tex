%! Author = Adrian Helberg
%! Date = 21.09.2020

% Preamble
\documentclass[11pt]{article}
% Packages
\usepackage{ngerman}
\usepackage{amsmath}
\usepackage{enumitem,amssymb}
\newlist{todolist}{itemize}{2}
\setlist[todolist]{label=$\square$}
\usepackage{pifont}
\newcommand{\cmark}{\ding{51}}%
\newcommand{\xmark}{\ding{55}}%
\newcommand{\done}{\rlap{$\square$}{\raisebox{2pt}{\large\hspace{1pt}\cmark}}%
\hspace{-2.5pt}}
\newcommand{\wontfix}{\rlap{$\square$}{\large\hspace{1pt}\xmark}}

\title{\textbf{Exposé} zur Bachelorarbeit von Adrian Helberg bei Prof. Dr. Jenke}

% Document
\begin{document}

    \maketitle

    \section{Problemstellung}
    Zu behandelndes Problem:
    \begin{itemize}
        \item Modellierung von Vegetation dauert lange
        \item Sieht nicht natürlich aus, wenn man immer nur das selbe Modell platziert
        \item Designer müssen Programmierkenntnisse haben, wenn sie auf prozedurale Porgrammierung setzen möchten
    \end{itemize}

    \section{Ziele der Arbeit}
    System programmieren zur Generierung von ``ähnlich aussehenden Modellen"' anhand eines Input-Modells

    \section{Ablauf}
    Zweiwöchige Meilensteine mit Besprechungen (Videokonferenz, Treffen)

    \section{Organisatorisches}
    \section{Voraussetzungen}

    \newpage

    \section{Zeitplan}
    \begin{todolist}
        \item[\done] bis 29.08.2020: Erarbeitung Basisquelle \cite{basisquelle}
        \item[\done] 14.09.2020: Kennenlerngespräch, erstes Themengespräch
        \item bis 28.09.2020: Version 0.1 des Exposés für die Bachelorarbeit
        \item bis 12.10.2020: Literaturrecherche, Mindmap der untersuchten Quellen
        \item bis 16.10.2020: Ausarbeitung des Software-/Hardwarestacks
        \item bis 26.10.2020: Erstellung des Projektplans
        \item ab 26.10.2020: Bearbeitung des praktischen Teils der Bachelorarbeit
    \end{todolist}

    ~\nocite{*}
    \bibliography{literature}
    \bibliographystyle{plain}

\end{document}