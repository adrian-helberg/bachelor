% @author Arian Helberg

\chapter{Grundlagen}

Die Modellierung mithilfe von Grafiksoftware ist eine vergleichbar händische, langwierige Erstellung von
Objekten.
Hierbei hat der Designer (Modellierer) die volle Kontrolle über die Strukturen des Objektes.\\
Bei der prozeduralen Modellierung werden spezifische Strukturen eines zu erstellenden physikalischen Objektes
generalisiert und meist über eine Grammatik und globale Parameter abgebildet.
Während bei der klassischen Modellierung die menschliche Intuition und bei der prozeduralen Modellierung eine
parametrisierte Grammatik genutzt werden kann, arbeitet die inverse prozedurale Modellierung mit bestehenden Modellen.
Sie extrahiert ("`lernt"') die Strukturen des Objekts, die automatisch in eine formale Grammatik überführt werden können.
Die Generierung von prozeduralen Modellen ist ein wichtiges, offenes Problem~\cite{benes_2011}.

\section{Grundbegriffe}

\subsection*{Modellierung}
Um einen physikalischen Körper in ein digitales Objekt zu überführen, wird mithilfe von Abstraktion (Modellierung)
ein mathematisches Modell erstellt, das diesen Körper formal beschreibt.
3D Grafiksoftware, wie bspw. Blender~\cite{blender}, wird genutzt um geometrische Körper zu modellieren, texturieren
und zu animieren.

\newpage

\subsection*{Prozedurale Modellierung}
\begin{quote}
    "`It encompasses a wide variety of generative techniques that
    can (semi-−)automatically produce a specific type of content based on a set of input
    parameters"'~\cite{smelik_2014}
\end{quote}
Prozedurale Modellierung beschreibt generative Techniken, die \\(semi-)automatisch spezifische, digitale
Inhalte anhand von deskriptiven Parametern erzeugen.
\citeauthor{smelik_2014} beschreibt einen Prozess, welcher durch das Nutzen globaler Parameter und descriptiver Regeln
Modelle erzeugt~\cite{smelik_2014}.

\subsection*{Inverse prozedurale Modellierung}
\citeauthor{aliaga_2016} spricht bei der inversen prozeduralen Modellierung von dem Finden einer prozeduralen Repräsentation
von Strukturen bestehender Modelle~\cite{aliaga_2016}.
Die Methodik aus Strukturen bestimmte Regeln und Parameter abzuleiten ist der Hauptgegenstand dieses Feldes der
Computergrafik und aktueller Gegenstand der Forschung.

\section{Grundlegende Arbeiten}

\citeauthor{smelik_2014} untersucht prozedurale Methoden, um diverse Strukturen, wie Vegetation, Straßen u.v.m zu erzeugen.
Es wird ein Überblick aktueller, vielversprechender Studien gegeben, deren Anwendung sowohl in technischen Bereichen,
als auch in nicht-technischen, kreativen Bereichen, diskutiert wird.
Diese Übersicht gilt als grundlegender Einstiegspunkt in die Bereiche der prozeduralen
Modellierung~\cite{smelik_2014}.
Einen aktuellen Stand der Forschung zur inversen prozeduralen Modellierung (IPM) liefert~\citeauthor{aliaga_2016}.
Es wird gezeigt, dass IPM-Ansätze in entsprechende Kernprobleme der Informatik aufgeteilt und meist getrennt voneinander durch
verschiedene Methodiken und Algorithmen bearbeitet werden~\cite{aliaga_2016}.
Weiter wird ein Einblick in die Kategorisierung und Bewertung einiger Ergebnisse von IPM-Systemem gegeben.
\citeauthor{higuera_2010} weist darauf hin, dass ein wesentlicher Unterschied zwischen der Induktion einer Grammatik
und der Grammatikinferierung besteht~\cite{higuera_2010}.
Die Induktion beschreibt das Finden einer Grammatik, welche ein Datum am genauesten beschreibt.
Die Grammatikinferierung ist das Finden einer Grammatik, die eine bestimmte Zeichenfolge abdeckt.

\newpage

\citeauthor{higuera_2010} zählt die inverse Generierung von L-Systemen zum Problem der Grammatikinferenz, welches er
als gut erforschtes Gebiet beschreibt.
Verzweigungsstrukturen im Kontext der inversen prozeduralen Modellierung tauchen in wissenschaftlichen Studien wenig auf.
~\citeauthor{guo_2020} adressiert einige IPM-Ansätze in seiner Arbeit~\cite{guo_2020}:\\~\\
Die Anwendung von modellierten Bäumen und Landschaften in den Bereichen wie Simulation, VR, Botanik und Architektur
wird in der Arbeit von \citeauthor{deussen_2010} gezeigt~\cite{deussen_2010}.
Hier liegt der Fokus auf der Erstellung und Kombination künstlicher Modelle, um natürlich wirkende Umgebungen zu
schaffen.\\~\\
Mit der Erkennung von Wiederholungsmustern in Hausfassaden beschäftigt sich\\~\citeauthor{alhalawani_2013}.
Durch das Finden diverser Verformungsparameter mithilfe einer faktorisierten Fassadendarstellung werden neue
Bildbearbeitungsmöglichkeiten  vorgestellt~\cite{alhalawani_2013}.\\~\\
Ein virtueller Wiederaufbau der archäologischen Stätte von Pompeji mithilfe der Erstellung von Gebäudemodellen wird
von~\citeauthor{mueller_2006} vorgestellt.\\~\\
Das Modellieren ganzer Städte über das \texttt{CityEngine}-System wird in der Arbeit von ~\citeauthor{parish_2001}
gezeigt.
Zunächst werden geografische Bilder eingelesen, aus welchen dann die Anordnung von Straßen, Gärten und Gebäuden
ausgelesen werden können.
Anschließend können virtuelle Städte nach dem Vorbild der Eingabe erstellt werden.\\~\\
Sowohl ~\citeauthor{merrell_2011}, als auch ~\citeauthor{zhang_2019} stellen Systeme zur Erzeugung von Inneneinrichtung
vor.
Mithilfe eines Layout-Systems werden Design-Patterns gelernt, in Terme einer Dichtefunktion überführt, um dann virtuelle
Möbel zu erstellen~\cite{merrell_2011}.
Eine Übersicht diverser Kriterien, die zur Erstellung von Innenraumszenen nötig sind, zeigt die
Wichtigkeit über die Auswahl sinnvoller Objekte und deren Anordnung, um plausible Räume zu erstellen~\cite{zhang_2019}.

\newpage

\subsection*{L-Systeme}
\citeauthor{lindemayer_1968} führt eine mathematische Beschreibung zum Wachstum fadenförmiger Organismen ein.
Sie zeigt, wie sich der Status von Zellen infolge ein oder mehrerer Einflüsse verhält~\cite{lindemayer_1968}.
Darüber hinaus führt er Systeme ein, die atomare Teile mithilfe von Produktionsregeln ersetzen (Ersetzungssysteme).
Diese L-Systeme (Lindenmayer-Systeme) nutzt er zur formalen Beschreibung von Zellteilung.
Später werden Symbole zur formalen Beschreibung von Verzweigungen, die von Filamenten abgehen, eingeführt~\cite{prusinkiewicz_1990}.
Die bekanntesten L-Systeme sind zeichenketten-basiert und werden von \textit{Noam Chomsky} eingeführt~\cite{chomsky_1956}.
Sie ersetzen parallel Symbole eines Wortes, welche von einer Grammatik über eine Sprache akzeptiert werden.
L-Systeme können unter anderem parametrisiert oder nicht-parametrisiert und kontextfrei oder kontextsensitiv sein.

L-Systeme sind Grammatiken mit folgender Form:
\begin{center}
    $\mathcal{L}=\langle M,\omega,R \rangle$, mit
    \begin{itemize}
        \item $M$ als Alphabet, das alle Symbole enthält, die in der Grammatik vorkommen,
        \item $\omega$ als Axiom oder "`Startwort"' und
        \item $R$ als Menge aller Produktionsregeln, die für $\mathcal{L}$ gelten
    \end{itemize}
\end{center}
Das Alphabet eines parametrisierten Systems enthält Module (Symbole mit Parametern) anstelle von Symbolen:
\begin{center}
    $M=\{A(P),B(P),\dots\}$ mit
    \begin{itemize}
        \item $P=p_1,p_2,\dots$ als Modulparameter
    \end{itemize}
\end{center}
Zeichen des Alphabets, die Ziel einer Produktionsregel sind, heißen Variablen oder Nonterminale.
Alle anderen Zeichen aus $M$ sind Konstanten oder Terminale.
Das Axiom $\omega$ ist eine nicht-leere Sequenz an Modulen aus $M^+$ mit
\begin{itemize}
    \item $M^+$ als Menge aller möglichen Zeichenketten aus Modulen aus $M$
\end{itemize}
Produktionsregeln sind geordnete Paare aus Wörtern über dem Alphabet, die bestimmte Ersetzungsregeln umsetzen.
Hierbei werden Symbole aus einem Wort, die einer linken Seite (\textit{engl. left hand side / LHS}) einer
Produktionsregel entsprechen, durch die rechte Seite (\textit{engl. right hand side / RHS}) ersetzt.
Sie sind folgendermaßen aufgebaut:
\begin{center}
    $A(P)\rightarrow x,x\in M^*$, mit
    \begin{itemize}
        \item $M^*$ als Menge aller möglichen Zeichenketten von M inklusive der leeren Zeichenkette $\varepsilon$.
    \end{itemize}
\end{center}
Ist die RHS jeder Produktionsregel ein einzelnes Symbol und gibt es zu jeder Variablen eine Regel, spricht man
von einem kontextfreien, andernfalls von einem kontextsensitiven L-System.

\subsubsection*{L-System Interpretation}
Lindenmayer-Systeme können Worte über ihr Alphabet interpretieren.
Hierzu werden Symbole des Wortes, die Ziel einer Produktionsregel sind, in Iterationen durch die RHS
der Produktionsregel ersetzt.
Bei der Ausführung eines L-Systems wird kein beliebiges Wort interpretiert, sondern das in der
Grammatik definierte Axiom.

\subsection*{Logo-Turtle-Algorithmus}
Der Logo-Turtle-Algorithmus~\cite{prusinkiewicz_1986} setzt ein Vorgehen zur graphischen Beschreibung von L-Systemen, bei dem
jeder Buchstabe in einem Wort einer bestimmten Zeichenoperation zugewiesen wird, um.
So kann aus einem L-System ein grafisches Muster generiert werden, das mit einer Abfolge von Zeichenbefehlen an
eine "`Schildkröte"' gezeichnet wird.
Das Triplett $(x,y,\theta)$ definiert den Status (State) der Schildkröte.
Dieser setzt sich aus der aktuellen Position $\left(\begin{smallmatrix} x \\ y \end{smallmatrix}\right)$ und dem
aktuellen Rotationswinkel $\theta$, der die Blickrichtung bestimmt, zusammen.\\
Der Algorithmus kann als Komprimierung eines geometrischen Musters gesehen werden.

\newpage

Symbole mit zugehörigen Steuerungsbefehlen und Statusveränderungen:
\begin{center}
    \begin{tabular}{lll}
        % ROW 1
        \textbf{Symbol} & \textbf{Steuerung} & \textbf{Statusveränderung} \\
        \hline \\
        % ROW 2
        $F(d)$ &
        \begin{minipage}{0.6\textwidth}
            Gehe vom derzeitigen Punkt $p_1$ $d$ Einheiten in die Blickrichtung zu dem Punkt $p_2$.
            Zeichne ein Liniensegment zwischen $p_1$ und $p_2$.\\
        \end{minipage} &
        ja
        \\ \hline \\
        % ROW 3
        $+(\alpha)$ &
        \begin{minipage}{0.6\textwidth}
            Setze neuen Rotationswinkel $\theta=\theta+\alpha$.\\
        \end{minipage} &
        ja
        \\ \hline \\
        % ROW 4
        $-(\alpha)$ &
        \begin{minipage}{0.6\textwidth}
            Setze neuen Rotationswinkel $\theta=\theta-\alpha$.\\
        \end{minipage} &
        ja
        \\ \hline \\
        % ROW 5
        $[$ &
        \begin{minipage}{0.6\textwidth}
            Lege den aktuellen State auf einen Stack.\\
        \end{minipage} &
        nein
        \\ \hline \\
        % ROW 6
        $]$ &
        \begin{minipage}{0.6\textwidth}
            Hole den State vom Stack.\\
        \end{minipage} &
        nein
    \end{tabular}
\end{center}

\section{Verwandte Arbeiten}
Die Erstellung von ähnlichen 3D-Objekten aus komplexen, dreidimensionalen Eingabe-objekten wird in der Arbeit von~\citeauthor{bokeloh_2010}
präsentiert~\cite{bokeloh_2010}.
Durch ein "`Zerschneiden"' des Eingabeobjektes in mehrere Unterobjekte, ist es möglich ähnliche Modelle aus den
Abschnitten zusammenzusetzen.
Hierbei werden Regeln zur Veränderung des Eingabeobjektes gefunden, die eine bestimmte lokale Ähnlichkeit beibehalten.
Alle benötigten Informationen werden direkt aus dem Modell und ohne Benutzerinteraktion abgeleitet.\\
Das Lernen von Design-Patterns mithilfe von Bayes-Grammatiken ist Gegenstand der Arbeit von~\citeauthor{talton_2012}.
Ein System zur Generierung geometrischer Modelle und\\Websites wird hier eingeführt, welches eine organisierte
Struktur von bezeichneten Teil-modellen entgegennimmt.
Über einen Prozess der MCMC-Optimierung wird eine Bayes-optimale Grammatik erstellt, um neue Modelle zu generieren~\cite{talton_2012}.
Das Markow-Chain-Monte-Carlo-Verfahren (MCMC-Verfahren) ist eine speicherlose, stichprobenartige Auswahl aus
Wahrscheinlichkeitsverteilungen eines definierten Bereichs, der für das Erreichen eines bestimmten Ziels von
Interesse ist.

\newpage

Auch~\citeauthor{stava_2014} etabliert einen MCMC-Ansatz zum Finden prozeduraler Repräsentationen für biologische Bäume.
Hier wird zunächst über eine Laplace-Glättung ein Grundgerüst gefunden, das dann in einer Baumtopologie organisiert wird~\cite{stava_2014}.\\
Das von ~\citeauthor{martinovic_2013} eingeführte System nutzt ähnlich organisierte Eingabestrukturen in Form von
Gebäudefassaden, um durch bayesische Grammatikinduktion eine kontextfreie Grammatik zu finden~\cite{martinovic_2013}.\\
Das Erstellen kontextfreier Grammatiken mithilfe statistischer Methoden zur Verteilung von zweidimensionalen Clustern
wird in einer Arbeit von~\citeauthor{stava_2010} gezeigt~\cite{stava_2010}.\\~\\
Die meisten Arbeiten im Bereich der inversen prozeduralen Modellierung beschäftigen sich eher mit der Rekonstruierung von
gegebenen Modellen, als mit der Generalisierung von Information.
Aus diesem Grund führt \citeauthor{guo_2020} ein Modell zum Lernen von L-Systemen von Verzweigungsstrukturen mithilfe maschinellen
Lernens (Deep Learning) anhand beliebiger Grafiken ein~\cite{guo_2020}.
Hierzu werden häufig genutzte Verzweigungen bezeichnet und in zufällige Verzweigungsstrukturen zusammengesetzt.
Mit diesen Trainings-daten wird ein neuronales Netz angelernt, sodass atomaren Verzweigungen in beliebigen Strukturen erkannt werden können.
In einer Eingabestruktur identifiziert das System die atomaren Strukturen und kennzeichnet diese mit einem Rechteck als Begrenzung.
Anhand von Überlappungen der Rechtecke kann dann eine hierarchische Baumtopologie aufgebaut werden, aus der ein kompaktes
L-System inferiert und generalisiert wird.
Das generalisierte L-System wird genutzt, um der Eingabestruktur ähnliche Strukturen zu erzeugen.\\~\\
Die Algorithmen zur Inferierung und Generalisierung von L-Systemen aus der Arbeit von~\citeauthor{guo_2020} sind
Gegenstand dieser Arbeit und werden für die Fragestellung adaptiert und implementiert.
Ein neuronales Netz zur Erkennung von Verzweigungsstrukturen liefert die vielversprechensten Ergebnisse bei der Analyse
solcher Strukturen.
Da die Implementierung eines neuronalen Netzes für diese Arbeit aus Praktikabilitätsgründen nicht möglich ist,
werden die für die Algorithmen benötigen Baumstrukturen während der benutzergesteuerten Erstellung einer Basisstruktur
erstellt.