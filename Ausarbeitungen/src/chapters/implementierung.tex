% @author Arian Helberg

\chapter{Implementierung}

\section{Pakete?}

Die Implementierung des Programms setzt sich aus folgenden Teilschritten zusammen:
\begin{itemize}
    \item Erstellung der \textit{GUI} (Paket gui, tree) mit
    \begin{itemize}
        \item UI-Elementen
        \item Render-Canvas
        \item Dateianbindung der Templates
        \item Erstellung der repräsentativen Baumstruktur\\ (\textit{treeGenerator}, Paket tree)
    \end{itemize}
    \item Implementierung der Subsysteme als Pipes
    \begin{itemize}
        \item \textit{Inferer} (Paket grammar): Ableiten eines kompakten L-Systems aus einer Baumstruktur
        \item \textit{Generalizer} (Paket grammar): Generieren eines generalisierten L-Systems anhand eines
        "`kleinen"' L-Systems
        \item \textit{Randomizer} (Paket grammar): Erzeugung von L-Systemen, die der erstellen
        Verzweigungsstruktur "`ähnlich"' sind
    \end{itemize}
    \item Komponenten- und Systemtests
\end{itemize}
\begin{center}
    \begin{tabular}{l|l}
        \textbf{Subsystem} & \textbf{Umsetzung} \\
        \hline \\
        GUI &
        \begin{minipage}[t]{0.8\textwidth}
            JavaFX als Framework zur Erstellung von grafischen und interaktiven Inhalten.
            Erstellung der Baumstruktur über dynamisches Erzeugen von Konten während der Strukturierung der
            Verzweigungsstruktur
        \end{minipage} \\
        \\ \hline \\
        Inferer &
        \begin{minipage}[t]{0.8\textwidth}
            Algorithmus zum Iterieren maximaler Sub-Bäume und deren Reduzierung mittels Ersetzung durch Symbole
            und der zugehörigen Produktionsregel, bis eine Kostengrenze, die durch eine Kostenfunktion abgebildet
            werden kann, erreicht ist
        \end{minipage} \\
        \\ \hline \\
        Generalizer &
        \begin{minipage}[t]{0.8\textwidth}
            Algorithmus zum Erweitern eines L-Systems um nicht-deterministischer Regeln und Erkennen rekursiver
            Strukturen
        \end{minipage} \\
        \\ \hline \\
        Randomizer &
        \begin{minipage}[t]{0.8\textwidth}
            ddd sfevdhbnreaydtydbfsdxc
        \end{minipage} \\
        \\ \hline
    \end{tabular}
\end{center}

\section{Entscheidungen}
\underline{Mutable or Immutable Objects?}\\~\\
\underline{Risiken}\\~\\
\underline{Qualitätsmerkmale}\\~\\
\underline{Alternativen}\\~\\
\underline{Aufwand der Implementierung}

\section{Technologien}

\begin{itemize}
    \item Programmiersprache: Java Version 11 mit
    \begin{itemize}
        \item JavaFX Version 15 (openjfx)
    \end{itemize}
    \item Build-Management-Tool: Gradle\cite{gradle} Version 6.7
    \item Versionskontrolle: Github Repository\cite{github} via Git\cite{git}
    \item IDE: JetBrains IntelliJ IDEA\cite{idea} 2020.2.2 (Ultimate Edition)
    \item Betriebssystem: Microsoft Windows 10 Pro 64 Bit
    \item User Story Map: Trello Board\cite{trello}
\end{itemize}

\section{Hardware}

\begin{itemize}
    \item Prozessor: Intel Core i5-3570K CPU @ 3.40GHz
\end{itemize}