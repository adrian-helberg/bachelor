% @author Arian Helberg

\chapter{Implementierung}

Zur Umsetzung der vorgestellten Konzepte wird im Folgenden auf Softwarepakete, Technologien, Datenspeicherung,
Benutzerinteraktion und Arbeitsablauf der erstellten Software eingegangen.
Darüber hinaus wird ein Überblick über genutzte Hardware und einige Implementierungsentscheidungen gegeben.
Auf Implementierungsdetails zur Nutzung der vorgestellten Technologien wird nicht eingegangen.

\section{Projektstruktur}
Um das Programm modular zu halten, werden die Softwarekomponenten in Pakete organisiert:


\section{Technologien}
\section{Fremdsysteme}
\section{User Interface}

Die Implementierung des Programms setzt sich aus folgenden Softwarepaketen zusammen:
\begin{itemize}
    \item[\SquareShadowTopLeft] \textit{de.haw}:
    \item[\SquareShadowTopLeft] \textit{de.haw.gui}:
    \item[\SquareShadowTopLeft] \textit{de.haw.lsystem}:
    \item[\SquareShadowTopLeft] \textit{de.haw.module}:
    \item[\SquareShadowTopLeft] \textit{de.haw.pipeline}:
    \item[\SquareShadowTopLeft] \textit{de.haw.tree}:
    \item[\SquareShadowTopLeft] \textit{de.haw.utils}:
\end{itemize}
\begin{center}
    \begin{tabular}{l|l}
        \textbf{Subsystem} & \textbf{Umsetzung} \\
        \hline \\
        GUI &
        \begin{minipage}[t]{0.8\textwidth}
            JavaFX als Framework zur Erstellung von grafischen und interaktiven Inhalten.
            Erstellung der Baumstruktur über dynamisches Erzeugen von Konten während der Strukturierung der
            Verzweigungsstruktur
        \end{minipage} \\
        \\ \hline \\
        Inferer &
        \begin{minipage}[t]{0.8\textwidth}
            Algorithmus zum Iterieren maximaler Sub-Bäume und deren Reduzierung mittels Ersetzung durch Symbole
            und der zugehörigen Produktionsregel, bis eine Kostengrenze, die durch eine Kostenfunktion abgebildet
            werden kann, erreicht ist
        \end{minipage} \\
        \\ \hline \\
        Generalizer &
        \begin{minipage}[t]{0.8\textwidth}
            Algorithmus zum Erweitern eines L-Systems um nicht-deterministischer Regeln und Erkennen rekursiver
            Strukturen
        \end{minipage} \\
        \\ \hline \\
        Randomizer &
        \begin{minipage}[t]{0.8\textwidth}
            ddd sfevdhbnreaydtydbfsdxc
        \end{minipage} \\
        \\ \hline
    \end{tabular}
\end{center}

\section{Entscheidungen}
\underline{Mutable or Immutable Objects?}\\~\\
\underline{Risiken}\\~\\
\underline{Qualitätsmerkmale}\\~\\
\underline{Alternativen}\\~\\
\underline{Aufwand der Implementierung}

\section{Technologien}

\begin{itemize}
    \item Programmiersprache: Java Version 11 mit
    \begin{itemize}
        \item JavaFX Version 15 (openjfx)
    \end{itemize}
    \item Build-Management-Tool: Gradle\cite{gradle} Version 6.7
    \item Versionskontrolle: Github Repository\cite{github} via Git\cite{git}
    \item IDE: JetBrains IntelliJ IDEA\cite{idea} 2020.2.2 (Ultimate Edition)
    \item Betriebssystem: Microsoft Windows 10 Pro 64 Bit
    \item User Story Map: Trello Board\cite{trello}
\end{itemize}

\section{Hardware}

\begin{itemize}
    \item Prozessor: Intel Core i5-3570K CPU @ 3.40GHz
\end{itemize}