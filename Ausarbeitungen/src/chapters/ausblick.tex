% @author Arian Helberg

\chapter{Fazit und Ausblick}
Diese Arbeit beschreibt die Konzeption und Implementierung eines prozeduralen Systems zur Erstellung
von Verzweigungsstrukturen, die einer Eingabestruktur ähneln.
Dabei werden Anforderungen untersucht, die zur Erstellung einer individuellen Software nötig sind.
Eine intuitiv nutzbare Anwendung zur benutzergesteuerten Strukturierung von eingelesenen Templates und
anschließender Generierung von Ähnlichkeitsstrukturen, ist im Rahmen dieser Arbeit entstanden.

\subsection*{Fazit}
Diese Arbeit zeigt, wie Ansätze der aktuellen Forschung genutzt werden können, um Verzweigungsstrukturen
zu erzeugen.
L-Systeme eignen sich gut, um Verzweigungsstrukturen mathematisch zu beschreiben und so in einem
Programm zu verwenden.
Sie können durch Ausführung und Interpretation mithilfe eines Logo-Turtle-Algorithmus visualisiert
und mittels grafischer Elemente sichtbar gemacht werden.
Die Steuerung von Gewichtungsparametern durch den Benutzer verhindert unnötige
\textit{Trial and Error}-Szenarien, da das Ergebnis des Prozesses gesteuert werden kann.
Die Erstellung und Integration eines neuronalen Netzes, das laut aktueller Forschung gute Ergebnisse beim Lernen von
Regeln aus einer Eingabestruktur liefert, kann in dieser Arbeit aus Praktikabilitätsgründen nicht behandelt werden.\\
Die umgesetzte Softwaretechnik zur Erstellung des Softwareprojekts stellt sich bedingt als praktikabel
heraus.
Auf der einen Seite ist ein gewisses Maß an Expertise in der Java-Spezifikation JavaFX nötig, um
die Implementierung mit einem test-driven Ansatz zu beginnen.
Weiter ist die Erreichung einer ausreichenden Testabdeckung mit Modultests nur schwer
zu erreichen, da einige Module eine lauffähige JavaFX-Instanz benötigen.
Auf der anderen Seite hilft sie eine angemessene Testabdeckung und so hochwertigen Quellcode
zu erzeugen.
Darüber hinaus ist das Softwareprojekt durch den Ansatz gut strukturiert und verringert
den Aufbau technischer Schuld.

\subsection*{Ausblick}
Die Eingabestrukturen lassen sich ohne erneute Konzeptionierung in andere Strukturen, die ebenfalls
einer Baumtopologie entsprechen, überführen.
L-Systeme beschreiben Baumstrukturen grundlegend und können diese durch deren Ausführung visualisieren.
So können bspw. Websites, die durch das \textit{DOM} spezifiziert sind, auf diese Weise interpretiert
und erstellt werden.
Hierzu muss die Software um eine Bereitstellung des erzeugten, generalisierten L-Systems erweitert
werden.\\
Eine Weiterführung des Softwareprojekts kann durch den Einsatz neuronaler Netze zum Lernen
struktureller Regeln und Ableiten von Transformationsparametern erreicht werden.
Die Erstellung von Ähnlichkeitsstrukturen lässt sich durch neuronale Netze automatisieren und vereinfachen,
was jedoch mit einem initialen Mehraufwand bezüglich der Implementierung verbunden ist.\\
Weiter lassen sich die Konzepte zur Synthese zweidimensionaler Strukturen in zukünftigen Arbeiten
in den dreidimensionalen Raum überführen.\\
Zum Schluss ist das Erkennen von Überlappungen von Strukturen ein schwieriges Problem der
Computergrafik und kann durch zusätzliche Algorithmen umgesetzt werden.