% @author Arian Helberg

\chapter{Fazit und Ausblick}
Diese Arbeit beschreibt die Konzeption und Implementierung eines prozeduralen Systems zur Erstellung
von Verzweigungsstrukturen, die einer Eingabestruktur ähneln.
Dabei werden Anforderungen untersucht, die zur Erstellung einer individuellen Software nötig sind.
Eine intuitiv nutzbare Anwendung zur benutzergesteuerten Strukturierung von eingelesenen Templates und
anschließender Generierung von Ähnlichkeitsstrukturen, ist im Rahmen dieser Arbeit entstanden.

\subsection*{Fazit}
L-Systeme eignen sich gut, um Verzweigungsstrukturen mathematisch zu beschreiben und so in einem
Programm zu verwenden.
Sie können durch Ausführung und Interpretation durch einen Logo-Turtle-Algorithmus visualisiert
und mittels grafischer Bedienelemente sichtbar gemacht werden.
Die Steuerung von Gewichtungsparametern durch den Benutzer verhindert unnötige
\textit{Trial and Error}-Szenarien, da das Ergebnis des Prozesses gesteuert werden kann.

\subsection*{Ausblick}
Die Eingabestrukturen lassen sich ohne erneute Konzeptionierung in andere Strukturen, die ebenfalls
einer Baumtopologie entsprechen, überführen.
L-Systeme beschreiben Baumstrukturen grundlegend und können diese durch deren Auflösung visualisieren.
So können bspw. Websites, die durch das \textit{DOM} spezifiziert sind, auf diese Weise interpretiert
und erstellt werden.
Hierzu muss die Software um eine Bereitstellung des erzeugten, generalisierten L-Systems erweitert
werden.\\
Eine Weiterführung des Softwareprojekts kann durch den Einsatz neuronaler Netze zum Lernen
struktureller Regeln und Ableiten von Transformationsparametern erreicht werden.
Auf der einen Seite lässt sich die Erstellung von Ähnlichkeitsstrukturen durch neuronale Netze
automatisieren und vereinfachen, auf der anderen Seite entsteht dadurch ein initialer Mehraufwand
der Implementierung.\\
Weiter lassen sich die Konzepte zur Synthese zweidimensionaler Strukturen in zukünftigen Arbeiten
in den dreidimensionalen Raum überführen.
Zum Schluss ist das Erkennen von Überlappungen von Strukturen ein schwieriges Problem der
Computergrafik und kann durch zusätzliche Algorithmen umgesetzt werden.